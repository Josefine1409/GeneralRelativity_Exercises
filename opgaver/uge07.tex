\documentclass[../main.tex]{subfiles}

\begin{document}

%%%%%%%%%%%%%%%%%%%%%%%%%%%%%%%%%%%%%%%%%%%%%%%%%%%%%%%%%%%%%%%%%%%%%%%%%%%%%%%%%%%%%

\section{Uge 7 -- Einsteinligningen og Riemanntensorer}
\setcounter{section}{7}


%%%%%%%%%%%%%%%%%%%%%%%%%%%%%%%%%%%%%%%%%%%%%%%%%%%%%%%%%%%%%%%%%%%%%%%%%%%%%%%%

\subsection{Opgave 1 -- Riemanntensor med sænket indeks}
\setcounter{subsection}{1}
\setcounter{equation}{0}

Bevis at Riemanntensoren med nedsænket indeks, $R_{abcd} \doteq g_{ae} R^{e}_{bcd}$, kan skrives som
\begin{align} \label{eq:Uge7_Opg1_RiemannTensorToShow_FirstPart}
    R_{abcd} &= - \Gamma_{abc,d} + \Gamma_{abd,c} + \Gamma_{eac} \Gamma^e_{bd} + \Gamma_{ead} \Gamma^e_{bc} \: ,
\end{align}
og som
\begin{align} \label{eq:Uge7_Opg1_RiemannTensorToShow_SecondPart}
    R_{abcd} &= \inv{2} \left( g_{ad,bc} + g_{bc,ad} - g_{ac,bd} - g_{bd,ac} \right) - \Gamma_{eac} \Gamma^e_{bd} + \Gamma_{ead} \Gamma^e_{bc} \: .
\end{align}

%%%%%%%%%%%%%%%%%%%%%%%%%

\subsubsection*{Besvarelse}

Før vi begynder at bevise ovenstående, da vil vi først bevise to identiteter
\begin{enumerate}[
    leftmargin=7em,
    label={Identitet \arabic*:},
    ref={Identitet \arabic*}
]
    \item
        \begin{align}
            \Gamma_{aed} &= - \Gamma_{ead} + g_{ae,d} \: , \quad \text{og}
        \end{align}
    \item
        \begin{align}
            g_{ae} \Gamma^e_{bc,d} &= \Gamma_{abc,d} - g_{ea,d} \Gamma^e_{bc} \: .
        \end{align}
\end{enumerate}
$ $\\

Først bevises identitet 1.\\
Fra $\mathrm{D}g_{ab} = 0$ har vi, at $g_{ab,c} = \Gamma_{bac} + \Gamma_{abc}$ (\cite[fodnote 8]{ugeseddel3}), hvorfor
\begin{align}
    g_{ae,d} &= \Gamma_{ead} + \Gamma_{aed}
        \quad \Rightarrow \quad
    \Gamma_{aed} = \Gamma_{ead} - g_{ae,d} \: .
\end{align}
\\

Dernæst bevises identitet 2.\\
Fra Leibniz' regel har vi, at $(AB)_{;b} = A_{;b}B + AB_{;b}$, hvorved
\begin{align}
\begin{split}
    \Gamma_{abc,d} &= \left( \Gamma_{abc} \right)_{,d}
        = \left( g_{ae} \Gamma^e_{bc} \right)_{,d}
        = g_{ae,d} \Gamma^e_{bc} + g_{ae} \Gamma^e_{bc,d} \\
    \Rightarrow g_{ae} \Gamma^e_{bc,d} &= \Gamma_{abc,d} - g_{ae,d} \Gamma^e_{bc} \: .
\end{split}
\end{align}
\\

Vi vil nu bevise første del af opgaven, nemlig \cref{eq:Uge7_Opg1_RiemannTensorToShow_FirstPart}.
\begin{align}
\begin{split}
    R_{abcd} &\doteq g_{ae} R^e_{bcd} \\
        &= g_{ae} \left( - \Gamma^e_{bc,d} + \Gamma^e_{bd,c} - \Gamma^f_{bc} \Gamma^e_{fd} + \Gamma^f_{bd} \Gamma^e_{fc} \right) \\
        &= - g_{ae} \Gamma^e_{bc,d} + g_{ae} \Gamma^e_{bd,c} - g_{ae} \Gamma^f_{bc} \Gamma^e_{fd} + g_{ae} \Gamma^f_{bd} \Gamma^e_{fc} \\
        &\xleq{2} - \left( \Gamma_{abc,d} - g_{ae} \Gamma^e_{bc} \right) + \left( \Gamma_{abd,c} - g_{ae} \Gamma^e_{bd} \right) - \Gamma^f_{bc} \Gamma_{afd} + \Gamma^f_{bd} \Gamma_{afc} \\
        &= - \Gamma_{abc,d} + g_{ae} \Gamma^e_{bc} + \Gamma_{abd,c} - g_{ae} \Gamma^e_{bd} - \Gamma^{\cancelto{e}{f}}_{bc} \Gamma_{a\cancelto{e}{f}d} + \Gamma^{\cancelto{e}{f}}_{bd} \Gamma_{a\cancelto{e}{f}c} \\
        &= - \Gamma_{abc,d} + g_{ae} \Gamma^e_{bc} + \Gamma_{abd,c} - g_{ae} \Gamma^e_{bd} - \Gamma^e_{bc} \Gamma_{aed} + \Gamma^e_{bd} \Gamma_{aec} \\
        &\xleq{1} - \Gamma_{abc,d} + g_{ae} \Gamma^e_{bc} + \Gamma_{abd,c} - g_{ae} \Gamma^e_{bd} - \Gamma^e_{bc} \left( - \Gamma_{ead} + g_{ae,d} \right) + \Gamma^e_{bd} \left( - \Gamma_{eac} + g_{ae,c} \right) \\
        &= - \Gamma_{abc,d} + \cancel{g_{ae} \Gamma^e_{bc}} + \Gamma_{abd,c} \cancel{- g_{ae} \Gamma^e_{bd}} + \Gamma^e_{bc} \Gamma_{ead} \cancel{- \Gamma^e_{bc} g_{ae,d}} - \Gamma^e_{bd} \Gamma_{eac} + \cancel{\Gamma^e_{bd} g_{ae,c}} \\
        &= - \Gamma_{abc,d} + \Gamma_{abd,c} - \Gamma_{eac} \Gamma^e_{bd} + \Gamma_{ead} \Gamma^e_{bc} \: .
\end{split}
\end{align}
Dermed er \cref{eq:Uge7_Opg1_RiemannTensorToShow_FirstPart} vist.
\\

Nu bevises anden del af opgave, nemlig \cref{eq:Uge7_Opg1_RiemannTensorToShow_SecondPart}.\\
Først bemærkes det, at
\begin{align}
    \Gamma_{abc} &= \inv{2} \left( g_{ab,c} - g_{bc,a} + g_{ac,b} \right) \: ,
\end{align}
\begin{subequations}
\begin{align}
    \Gamma_{abc,d} &= \inv{2} \left( g_{ab,c} - g_{bc,a} + g_{ac,b} \right)_{,d}
        = \inv{2} \left( g_{ab,cd} - g_{bc,ad} + g_{ac,bd} \right) \: , \quad \text{og} \\
    \Gamma_{abd,c} &= \inv{2} \left( g_{ab,d} - g_{bd,a} + g_{ad,b} \right)_{,c}
        = \inv{2} \left( g_{ab,dc} - g_{bd,ac} + g_{ad,bc} \right) \: .
\end{align}
\end{subequations}
Dermed bliver de første to led af \cref{eq:Uge7_Opg1_RiemannTensorToShow_FirstPart}
\begin{align}
\begin{split}
    - \Gamma_{abc,d} + \Gamma_{abd,c} &= \inv{2} \left( -g_{ab,cd} + g_{bc,ad} - g_{ac,bd} + g_{ab,dc} - g_{bd,ac} + g_{ad,bc} \right) \\
        &= \inv{2} \left( \cancel{-g_{ab,cd}} + g_{bc,ad} - g_{ac,bd} + \cancel{g_{ab,dc}} - g_{bd,ac} + g_{ad,bc} \right) \\
        &= \inv{2} \left( g_{ad,bc} + g_{bc,ad} - g_{ac,bd} - g_{bd,ac} \right) \\
\end{split}
\end{align}

Dermed bliver
\begin{align}
\begin{split}
    R_{abcd} &= - \Gamma_{abc,d} + \Gamma_{abd,c} - \Gamma_{eac} \Gamma^e_{bd} + \Gamma_{ead} \Gamma^e_{bc} \\
        &= \inv{2} \left( g_{ad,bc} + g_{bc,ad} - g_{ac,bd} - g_{bd,ac} \right) - \Gamma_{eac} \Gamma^e_{bd} + \Gamma_{ead} \Gamma^e_{bc} \: ,
\end{split}
\end{align}
og anden del af opgaven er vist.



%%%%%%%%%%%%%%%%%%%%%%%%%%%%%%%%%%%%%%%%%%%%%%%%%%%%%%%%%%%%%%%%%%%%%%%%%%%%%%%%

\subsection{(M) Opgave 2 -- Bevis for eksistens af koordinatsystem med alle Christoffelsymboler værende nul}
\setcounter{subsection}{2}
\setcounter{equation}{0}

Bevis at det altid er muligt at vælge et koordinatsystem i hvilket alle Christoffelsymboler er nul i et givent punkt.

%%%%%%%%%%%%%%%%%%%%%%%%%

\subsubsection{Besvarelse}

\ldots



%%%%%%%%%%%%%%%%%%%%%%%%%%%%%%%%%%%%%%%%%%%%%%%%%%%%%%%%%%%%%%%%%%%%%%%%%%%%%%%%

\subsection{(M) Opgave 3 -- Christoffelsymboler nul i punkt medfører førsteordensafledede af metrisk tensor værende nul}
\setcounter{subsection}{3}
\setcounter{equation}{0}

Vis at hvis alle Christoffelsymboler i et givent punkt er nul, så er førsteordensafledede (men ikke nødvendigvis andenordensafledede) af den metriske tensor også nul.

Hint: Udtryk de afledede af den metriske tensor som linearkombinationer af Christoffelsymboler.

%%%%%%%%%%%%%%%%%%%%%%%%%

\subsubsection{Besvarelse}

\ldots



%%%%%%%%%%%%%%%%%%%%%%%%%%%%%%%%%%%%%%%%%%%%%%%%%%%%%%%%%%%%%%%%%%%%%%%%%%%%%%%%

\subsection{Opgave 4 -- Ombytningssymmetri af Riemanntensor}
\setcounter{subsection}{4}
\setcounter{equation}{0}

Bevis ombytningssymmetrien (eng: interchange symmetry) af Riemanntensoren,
\begin{align}
    R_{abcd} &= R_{cdab} \: .
\end{align}

%%%%%%%%%%%%%%%%%%%%%%%%%

\subsubsection*{Besvarelse}

Vi skal vise, at Riemanntesoren er symmetrisk under ombytning af indeks, hvilket er egenskab 2 i afsnittet om ''Properties of the curvature tensor'' på ugeseddel 7 (\cite[lign. 13]{ugeseddel7}). Dette gøres ved at benytte to af de andre egenskaber ved tensoren, nemlig \cite[lign. 12 og 14]{ugeseddel7}
\begin{enumerate}[
    leftmargin=7em,
    label={Egenskab \arabic*:},
    ref={Egenskab \arabic*}
]
    \item Antisymmetri
        \begin{align}
            R_{abcd} &= - R_{bacd} = - R_{abdc} \: , \quad \text{og}
        \end{align}
    \setcounter{enumi}{2}
    \item Algebraisk Bianchi identitet
        \begin{align}
            0 &= R_{abcd} + R_{acdb} + R_{adbc} \: .
        \end{align}
\end{enumerate}

Vi kan nu bevise egenskab 2, altså ombytningssymmetrien af Riemanntensoren
\begin{align}
\begin{split}
    R_{abcd} &\xleq{3} - R_{acdb} - R_{adbc} \\
        &\xleq{1} R_{cadb} + R_{dabc} \\
        &\xleq{3} - R_{cdba} - R_{cbad} - R_{dbca} - R_{dcab} \\
        &\xleq{1} R_{cdab} + R_{bcad} + R_{bdca} + R_{cdab} \\
        &= 2 R_{cdab} + R_{bcad} + R_{bdca} \\
        &\xleq{3} 2 R_{cdab} - R_{badc} \\
        &\xleq{1} 2 R_{cdab} + R_{abdc} \\
        &\xleq{1} 2 R_{cdab} - R_{abcd} \\
    \Rightarrow 2 R_{abcd} &= 2 R_{cdab} \\
    \Rightarrow R_{abcd} &= R_{cdab} \: .
\end{split}
\end{align}
Dermed er ombytningssymmetrien af Riemanntensoren vist.



%%%%%%%%%%%%%%%%%%%%%%%%%%%%%%%%%%%%%%%%%%%%%%%%%%%%%%%%%%%%%%%%%%%%%%%%%%%%%%%%

\subsection{(M) Opgave 5 -- Bevis antisymmetri af Riemanntensoren}
\setcounter{subsection}{5}
\setcounter{equation}{0}

Bevis antisymmetrien af Riemanntensoren.

%%%%%%%%%%%%%%%%%%%%%%%%%

\subsubsection{Besvarelse}

\ldots



%%%%%%%%%%%%%%%%%%%%%%%%%%%%%%%%%%%%%%%%%%%%%%%%%%%%%%%%%%%%%%%%%%%%%%%%%%%%%%%%

\subsection{Opgave 6 -- Bevis af den algebraisk Bianchiidentitet}
\setcounter{subsection}{6}
\setcounter{equation}{0}

Bevis den algebraiske Bianchiindentitet.

%%%%%%%%%%%%%%%%%%%%%%%%%

\subsubsection{Besvarelse}

Til beviset af den algebraiske Bianchiidentitet af Riemanntensoren,
\begin{align}
    0 &= R_{abcd} + R_{acdb} + R_{adbc} \: ,
\end{align}
benyttes at Riemanntensoren er givet ved
\begin{align}
    R^a_{bcd} &= \Gamma^a_{bd,c} - \Gamma^a_{bc,d} + \Gamma^a_{ec} \Gamma^e_{bd} - \Gamma^a_{ed} \Gamma^e_{bc} \: ,
\end{align}
og at $\Gamma_{abc} = \Gamma_{acb}$.

Vi får derved, at
\begin{align}
\begin{split}
    R_{abcd} + R_{acdb} + R_{adbc} &= g_{af} \left[ R^f_{bcd} + R^f_{cdb} + R^f_{dbc} \right] \\
        &= g_{af} \Big[ \left\{ \Gamma^f_{bd,c} - \Gamma^f_{bc,d} + \Gamma^f_{ec} \Gamma^e_{bd} - \Gamma^f_{ed} \Gamma^e_{bc} \right\} \\
            &\qquad\quad + \left\{ \Gamma^f_{cb,d} - \Gamma^f_{cd,b} + \Gamma^f_{ed} \Gamma^e_{cb} - \Gamma^f_{eb} \Gamma^e_{cd} \right\} \\
            &\qquad\quad + \left\{ \Gamma^f_{dc,b} - \Gamma^f_{db,c} + \Gamma^f_{eb} \Gamma^e_{dc} - \Gamma^f_{ec} \Gamma^e_{db} \right\} \Big] \\
        &= g_{af} \Big[
                \Gamma^f_{cb,d} - \Gamma^f_{bc,d}
                + \Gamma^f_{bd,c} - \Gamma^f_{db,c}
                + \Gamma^f_{dc,b} - \Gamma^f_{cd,b} \\
                &\qquad\quad + \Gamma^f_{ed} \left\{ \Gamma^e_{cb} - \Gamma^e_{bc} \right\}
                + \Gamma^f_{ec} \left\{ \Gamma^e_{bd} - \Gamma^e_{db} \right\}
                + \Gamma^f_{eb} \left\{ \Gamma^e_{dc} - \Gamma^e_{cd} \right\}
            \Big] \\
        &= g_{af} \Big[
                \Gamma^f_{bc,d} - \Gamma^f_{bc,d}
                + \Gamma^f_{bd,c} - \Gamma^f_{bd,c}
                + \Gamma^f_{cd,b} - \Gamma^f_{cd,b} \\
                &\qquad\quad + \Gamma^f_{ed} \left\{ \Gamma^e_{bc} - \Gamma^e_{bc} \right\}
                + \Gamma^f_{ec} \left\{ \Gamma^e_{bd} - \Gamma^e_{bd} \right\}
                + \Gamma^f_{eb} \left\{ \Gamma^e_{cd} - \Gamma^e_{cd} \right\}
            \Big] \\
        &= \g_{af} \cdot 0 \\
        &= 0 \: .
\end{split}
\end{align}
Dermed er den algebraiske Bianchiidentitet af Riemanntensoren bevist,
\begin{align}
    0 &= R_{abcd} + R_{acdb} + R_{adbc} \: .
\end{align}



%%%%%%%%%%%%%%%%%%%%%%%%%%%%%%%%%%%%%%%%%%%%%%%%%%%%%%%%%%%%%%%%%%%%%%%%%%%%%%%%

\subsection{(M) Opgave 7 -- Beivs af den differentielle Bianchiidentitet}
\setcounter{subsection}{7}
\setcounter{equation}{0}

Bevis den differentielle Bianchiidentitet.

%%%%%%%%%%%%%%%%%%%%%%%%%

\subsubsection{Besvarelse}

\ldots



%%%%%%%%%%%%%%%%%%%%%%%%%%%%%%%%%%%%%%%%%%%%%%%%%%%%%%%%%%%%%%%%%%%%%%%%%%%%%%%%

\subsection{(M) Opgave 8 -- Bevis $(R^{ab} - Rg^{ab}/2)_{;a} = 0$}
\setcounter{subsection}{8}
\setcounter{equation}{0}

Bevis at $(R^{ab} - Rg^{ab}/2)_{;a} = 0$.

%%%%%%%%%%%%%%%%%%%%%%%%%

\subsubsection{Besvarelse}

\ldots



%%%%%%%%%%%%%%%%%%%%%%%%%%%%%%%%%%%%%%%%%%%%%%%%%%%%%%%%%%%%%%%%%%%%%%%%%%%%%%%%

\subsection{Opgave 9 -- Riemanntensor for $\dd s^2 = \dd r^2 + r^2 \dd \phi^2$}
\setcounter{subsection}{9}
\setcounter{equation}{0}

Beregn Riemanntensoren for rummet med metrikken $\dd s^2 = \dd r^2 + r^2 \dd \phi^2$ og diskuter resultatet.

%%%%%%%%%%%%%%%%%%%%%%%%%

\subsubsection{Besvarelse}

Riemanntensoren er givet som
\begin{align}
    R^a_{bcd} &= - \Gamma^a_{bc,d} + \Gamma^a_{bd,c} - \Gamma^e_{bc} \Gamma^a_{ed} + \Gamma^e_{bd} \Gamma^a_{ec} \: ,
\end{align}
hvor vi for den opgivne metrik har at $a,b,c,d,e=r,\theta$.

Fra \textbf{opgave 6 i uge 3} ved vi, at
\begin{align}
    g_{ab} &= \TwoRowMat{1 & 0}{0 & r^2}
        \quad \text{og} \quad
    g^{ab} = \TwoRowMat{1 & 0}{0 & \inv{r^2}} \: ,
    \qquad a,b=r,\theta \: ,
\end{align}
hvormed
\begin{subequations}
\begin{align}
    \Gamma^r_{\theta r} &= \inv{2} g^{rr} \left( g_{r\theta,r} - g_{\theta r,r} + g_{rr,\theta} \right)
        = \inv{2} g^{rr} g_{rr,\theta}
        = \inv{2} \cdot 1 \cdot \pdif{1}{x^\theta}
        = 0 \: , \\
    \Gamma^r_{\theta \theta} &= \inv{2} g^{rr} \left( g_{r\theta,\theta} - g_{\theta \theta,r} + g_{\theta r,\theta} \right)
        = \inv{2} \cdot 1 \cdot \left( 0 - \pdif{r^2}{r} + 0 \right)
        = - r \: , \quad \text{og} \\
    \Gamma^\theta_{\theta r} &= \inv{2} g^{\theta\theta} \left( g_{\theta\theta,r} - g_{\theta r,\theta} + g_{r\theta,\theta} \right)
        = \inv{2} g^{\theta\theta} g_{\theta\theta,r}
        = \inv{2} \inv{r^2} \pdif{r^2}{r}
        = \inv{r} \: .
\end{align}
\end{subequations}
Indsættes dette i Riemanntensoren fås
\begin{align}
\begin{split}
    R^r_{\theta r \theta} &= - \Gamma^r_{\theta r,\theta} + \Gamma^r_{\theta \theta,r} - \Gamma^e_{\theta r} \Gamma^r_{e \theta} + \Gamma^e_{\theta \theta} \Gamma^r_{er} \\
        &= - 0 + \pdif{}{r}(-r) - \left[ 0 + \inv{r} (-r) \right] + \left[ 0 + 0 \right] \\
        &= 0 \: .
\end{split}
\end{align}
Ved brug af antisymmetri og ombytningssymmetri af Riemanntensoren kan vi konkludere, at $R^a_{bcd} = 0$ for denne metrik, hvilket medfører at metrikken af de sfæriske koordinater i det todimensionelle Euklidiske rum er flad, hvilket også giver god mening.



%%%%%%%%%%%%%%%%%%%%%%%%%%%%%%%%%%%%%%%%%%%%%%%%%%%%%%%%%%%%%%%%%%%%%%%%%%%%%%%%

\subsection{Opgave 10 -- Riemanntensor, Riccitensor og Ricciskalar for todimensionel sfære}
\setcounter{subsection}{10}
\setcounter{equation}{0}

Beregn alle ikkeforsvindende komponenter af Riemanntensoren $R_{abcd}$, hvor hver af indeksene $a,b,c,d$ kan antage værdierne $\theta,\phi$, for metrikken
\begin{align}
    \dd s^2 &= r^2 \left[ \dd \theta^2 + \sin^2(\theta) \dd \phi^2 \right]
\end{align}
på en todimensionel sfære af radius $r$. Beregn også Riccitensoren $R_{ab}$ og skalarkrumningen (Ricciskalaren) $R$.

Svar:
\begin{align}
    R_{\theta\phi\theta\phi} &= r^2 \sin^2(\theta)
        = R_{\phi\theta\phi\theta}
        = - R_{\theta\phi\phi\theta}
        = - R_{\phi\theta\theta\phi} \: .
\end{align}

%%%%%%%%%%%%%%%%%%%%%%%%%

\subsubsection{Besvarelse}

For metrikken
\begin{align}
    \dd s^2 &= r^2 \left[ \dd \theta^2 + \sin^2(\theta) \dd \phi^2 \right]
\end{align}
har vi, at
\begin{align}
    g_{\theta\theta} = r^2 \: , \quad \text{og} \quad
    g_{\phi\phi} = r^2 \sin^2(\theta) \: ,
\end{align}
mens de andre kominationer af $g_{ab} = 0$, da $\dd s^2 = \sum_{a,b} g_{ab}\, \dd x^a\, \dd x^b$.

Da vi skal beregne Riemanntensoren, som består af Christoffelsymboler, da beregner vi først disse, hvilket gøres ved
\begin{align}
    \Gamma_{abc} &= \inv{2} \left[ g_{ab,c} - g_{bc,a} + g_{ca,b} \right] \: .
\end{align}
Christoffelsymbolerne bliver derved
\begin{subequations}
\begin{align}
    %%%%%%%%%%%%%%%%%%%%%%%%%%%%%%%%%%%%%%%%%%%%%%%%%%
    % a = \theta
    %%%%%%%%%%%%%%%%%%%%%%%%%%%%%%%%%%%%%%%%%%%%%%%%%%
    \begin{split}
        \Gamma^\theta_{\theta\theta} &= g^{\theta\theta} \Gamma_{\theta\theta\theta}
            = \frac{g^{\theta\theta}}{2} \left[ g_{\theta\theta,\theta} - g_{\theta\theta,\theta} + g_{\theta\theta,\theta} \right]
            = \frac{g^{\theta\theta}}{2} g_{\theta\theta,\theta}
            = \inv{2 r^2} \pdif{r^2}{\theta} \\
            &= 0 \: ,
    \end{split} \\
    \begin{split}
        \Gamma^\theta_{\phi\phi} &= g^{\theta\theta} \Gamma_{\theta\phi\phi}
            = \frac{g^{\theta\theta}}{2} \left[ g_{\theta\phi,\phi} - g_{\phi\phi,\theta} + g_{\phi\theta,\phi} \right]
            = \frac{g^{\theta\theta}}{2} \left[ - g_{\phi\phi,\theta} \right]
            = \frac{-1}{2r^2} \pdif{[r^2\sin^2(\theta)]}{\theta} \\
            &= - \sin(\theta) \cos(\theta) \: ,
    \end{split} \\
    \begin{split}
        \Gamma^\theta_{\theta\phi} &= g^{\theta\theta} \Gamma_{\theta\theta\phi}
            = \frac{g^{\theta\theta}}{2} \left[ g_{\theta\theta,\phi} - g_{\theta\phi,\theta} + g_{\phi\theta,\theta} \right]
            = \frac{g^{\theta\theta}}{2} g_{\theta\theta,\phi}
            = \inv{2r^2} \pdif{r^2}{\phi} \\
            &= 0 \: ,
    \end{split} \\
    %%%%%%%%%%%%%%%%%%%%%%%%%%%%%%%%%%%%%%%%%%%%%%%%%%
    % a = \phi
    %%%%%%%%%%%%%%%%%%%%%%%%%%%%%%%%%%%%%%%%%%%%%%%%%%
    \begin{split}
        \Gamma^\phi_{\theta\theta} &= g^{\phi\phi} \Gamma_{\pi\theta\theta}
            = \frac{g^{\phi\phi}}{2} \left[ g_{\phi\theta,\theta} - g_{\theta\theta,\phi} + g_{\theta\phi,\theta} \right]
            = \frac{g^{\phi\phi}}{2} \left[ - g_{\theta\theta,\phi} \right]
            = \frac{-1}{2r^2\sin^2(\theta)} \pdif{r^2}{\phi} \\
            &= 0 \: ,
    \end{split} \\
    \begin{split}
        \Gamma^\phi_{\phi\phi} &= g^{\phi\phi} \Gamma_{\phi\phi\phi}
            = \frac{g^{\phi\phi}}{2} \left[ g_{\phi\phi,\phi} - g_{\phi\phi,\phi} + g_{\phi\phi,\phi} \right]
            = \frac{g^{\phi\phi}}{2} g_{\phi\phi,\phi}
            = \inv{2r^2\sin^2(\theta)} \pdif{[r^2\sin^2(\theta)]}{\phi} \\
            &= 0 \: ,
    \end{split} \\
    \begin{split}
        \Gamma^\phi_{\phi\theta} &= g^{\phi\phi} \Gamma_{\phi\phi\theta}
            = \frac{g^{\phi\phi}}{2} \left[ g_{\phi\phi,\theta} - g_{\phi\theta,\phi} + g_{\theta\phi,\phi} \right]
            = \frac{g^{\phi\phi}}{2} g_{\phi\phi,\theta}
            = \inv{2r^2\sin^2(\theta)} \pdif{[r^2\sin^2(\theta)]}{\theta} \\
            &= \frac{\cos(\theta)}{\sin(\theta)}
            = \cot(\theta) \: .
    \end{split}
\end{align}
\end{subequations}
Yderligere vides det, at $\Gamma_{abc} = \Gamma_{acb}$, hvilket giver de resterende Christoffelsymboler.

Riemanntensoren er nu givet ved Christoffelsymbolerne ved
\begin{align}
    R^a_{bcd} &= \Gamma^a_{bd,c} - \Gamma^a_{bc,d} + \Gamma^a_{ec} \Gamma^e_{bd} - \Gamma^a_{ed} \Gamma^e_{bc}
\end{align}
hvorfor vi får nul for alle komponenter af Riemanntensoren med $c = d$, $\Gamma^a_{bcc} = 0$,
\begin{align}
    0 &= R_{\theta\theta\theta\theta}
        = R_{\theta\theta\phi\phi}
        = R_{\theta\phi\theta\theta}
        = R_{\theta\phi\phi\phi}
        = R_{\phi\theta\theta\theta}
        = R_{\phi\theta\phi\phi}
        = R_{\phi\phi\theta\theta}
        = R_{\phi\phi\phi\phi} \: ,
\end{align}
og de resterende komponenter af Riemanntensore giver
\begin{subequations}
\begin{align}
    %%%%%%%%%%%%%%%%%%%%%%%%%%%%%%%%%%%%%%%%%%%%%%%%%%
    % a = \theta, b = \theta
    %%%%%%%%%%%%%%%%%%%%%%%%%%%%%%%%%%%%%%%%%%%%%%%%%%
    % \begin{split}
    %     R^\theta_{\theta \theta \theta} &= \Gamma^\theta_{\theta \theta,\theta} - \Gamma^\theta_{\theta \theta,\theta} + \Gamma^\theta_{e\theta} \Gamma^e_{\theta \theta} - \Gamma^\theta_{e\theta} \Gamma^e_{\theta \theta} \\
    %         &= 0 \: ,
    % \end{split} \\
    \begin{split}
        R_{\theta\theta\theta\phi} &= g_{\theta\theta} R^\theta_{\theta \theta \phi} \\
            &= g_{\theta\theta} \left\{ \Gamma^\theta_{\theta \phi,\theta} - \Gamma^\theta_{\theta \theta,\phi} + \Gamma^\theta_{e\theta} \Gamma^e_{\theta \phi} - \Gamma^\theta_{e\phi} \Gamma^e_{\theta \theta} \right\} \\
            &= g_{\theta\theta} \left\{ \Gamma^\theta_{\theta \phi,\theta} - \Gamma^\theta_{\theta \theta,\phi} + \left[ \Gamma^\theta_{\theta\theta} \Gamma^\theta_{\theta \phi} + \Gamma^\theta_{\phi\theta} \Gamma^\phi_{\theta \phi} \right] - \left[ \Gamma^\theta_{\theta\phi} \Gamma^\theta_{\theta \theta} + \Gamma^\theta_{\phi\phi} \Gamma^\phi_{\theta \theta} \right] \right\} \\
            &= 0 \: ,
    \end{split} \\
    \begin{split}
        R_{\theta\theta\phi\theta} &\xleq{*} - R_{\theta\theta\theta\phi}
            = 0 \: ,
    \end{split} \\
    % \begin{split}
    %     R^\theta_{\theta \phi \phi} &= \Gamma^\theta_{\theta \phi,\phi} - \Gamma^\theta_{\theta \phi,\phi} + \Gamma^\theta_{e\phi} \Gamma^e_{\theta \phi} - \Gamma^\theta_{e\phi} \Gamma^e_{\theta \phi} \\
    %         &= 0 \: ,
    % \end{split} \\
    %%%%%%%%%%%%%%%%%%%%%%%%%%%%%%%%%%%%%%%%%%%%%%%%%%
    % a = \theta, b = \phi
    %%%%%%%%%%%%%%%%%%%%%%%%%%%%%%%%%%%%%%%%%%%%%%%%%%
    % \begin{split}
    %     R^\theta_{\phi \theta \theta} &= \Gamma^\theta_{\phi \theta,\theta} - \Gamma^\theta_{\phi \theta,\theta} + \Gamma^\theta_{e\theta} \Gamma^e_{\phi \theta} - \Gamma^\theta_{e\theta} \Gamma^e_{\phi \theta} \\
    %         &=  0 \: ,
    % \end{split} \\
    \begin{split}
        R_{\theta\phi\theta\phi} &= g_{\theta\theta} R^\theta_{\phi \theta \phi} \\
            &= g_{\theta\theta} \left\{ \Gamma^\theta_{\phi \phi,\theta} - \Gamma^\theta_{\phi \theta,\phi} + \Gamma^\theta_{e\theta} \Gamma^e_{\phi \phi} - \Gamma^\theta_{e\phi} \Gamma^e_{\phi \theta} \right\} \\
            &= g_{\theta\theta} \left\{ \Gamma^\theta_{\phi \phi,\theta} - \Gamma^\theta_{\phi \theta,\phi} + \left[ \Gamma^\theta_{\theta\theta} \Gamma^\theta_{\phi \phi} + \Gamma^\theta_{\phi\theta} \Gamma^\phi_{\phi \phi} \right] - \left[ \Gamma^\theta_{\theta\phi} \Gamma^\theta_{\phi \theta} + \Gamma^\theta_{\phi\phi} \Gamma^\phi_{\phi \theta} \right] \right\} \\
            &= g_{\theta\theta} \left\{ \Gamma^\theta_{\phi \phi,\theta} - \Gamma^\theta_{\phi\phi} \Gamma^\phi_{\phi \theta} \right\} \\
            &= r^2 \left\{ \left[ - \sin(\theta) \cos(\theta) \right]_{,\theta} - \left[ - \sin(\theta) \cos(\theta) \right] \pfrac{\cos(\theta)}{\sin(\theta)} \right\} \\
            &= r^2 \left\{ - \left[ \cos^2(\theta) - \sin^2(\theta) \right] + \cos^2(\theta) \right\} \\
            &= r^2 \sin^2(\theta) \: ,
    \end{split} \\
    \begin{split}
        R_{\theta\phi\phi\theta} &\xleq{*} - R_{\theta\phi\theta\phi}
            = - r^2 \sin^2(\theta) \: ,
    \end{split} \\
    % \begin{split}
    %     R^\theta_{\phi \phi \phi} &= \Gamma^\theta_{\phi \phi,\phi} - \Gamma^\theta_{\phi \phi,\phi} + \Gamma^\theta_{e\phi} \Gamma^e_{\phi \phi} - \Gamma^\theta_{e\phi} \Gamma^e_{\phi \phi} \\
    %         &= 0 \: ,
    % \end{split} \\
    %%%%%%%%%%%%%%%%%%%%%%%%%%%%%%%%%%%%%%%%%%%%%%%%%%
    % a = \phi, b = \theta
    %%%%%%%%%%%%%%%%%%%%%%%%%%%%%%%%%%%%%%%%%%%%%%%%%%
    % \begin{split}
    %     R^\phi_{\theta \theta \theta} &= \Gamma^\phi_{\theta \theta,\theta} - \Gamma^\phi_{\theta \theta,\theta} + \Gamma^\phi_{e\theta} \Gamma^e_{\theta \theta} - \Gamma^\phi_{e\theta} \Gamma^e_{\theta \theta}
    % \end{split} \\
    \begin{split}
        R_{\phi\theta\theta\phi} &\xleq{*} - R_{\theta\phi\theta\phi}
            = - r^2 \sin^2(\theta) \: ,
    \end{split} \\
    \begin{split}
        R_{\phi\theta\phi\theta} &\xleq{*} - R_{\phi\theta\theta\phi}
            = - \left( - R_{\theta\phi\theta\phi} \right)
            \xleq{*} R_{\theta\phi\theta\phi} \:: ,
    \end{split} \\
    % \begin{split}
    %     R^\phi_{\theta \phi \phi} &= \Gamma^\phi_{\theta \phi,\phi} - \Gamma^\phi_{\theta \phi,\phi} + \Gamma^\phi_{e\phi} \Gamma^e_{\theta \phi} - \Gamma^\phi_{e\phi} \Gamma^e_{\theta \phi}
    % \end{split} \\
    %%%%%%%%%%%%%%%%%%%%%%%%%%%%%%%%%%%%%%%%%%%%%%%%%%
    % a = \phi, b = \phi
    %%%%%%%%%%%%%%%%%%%%%%%%%%%%%%%%%%%%%%%%%%%%%%%%%%
    % \begin{split}
    %     R^\phi_{\phi \theta \theta} &= \Gamma^\phi_{\phi \theta,\theta} - \Gamma^\phi_{\phi \theta,\theta} + \Gamma^\phi_{e\theta} \Gamma^e_{\phi \theta} - \Gamma^\phi_{e\theta} \Gamma^e_{\phi \theta}
    % \end{split} \\
    \begin{split}
        R_{\phi \phi \theta \phi} &= g_{\phi\phi} R^\phi_{\phi \theta \phi} \\
            &= g_{\phi\phi} \left\{ \Gamma^\phi_{\phi \phi,\theta} - \Gamma^\phi_{\phi \theta,\phi} + \Gamma^\phi_{e\theta} \Gamma^e_{\phi \phi} - \Gamma^\phi_{e\phi} \Gamma^e_{\phi \theta} \right\} \\
            &= g_{\phi\phi} \left\{ - \Gamma^\phi_{\phi \theta,\phi} \right\} \\
            &= - r^2 \pfrac{\cos(\theta)}{\sin(\theta)}_{,\phi} \\
            &= 0 \: ,
    \end{split} \\
    \begin{split}
        R_{\phi \phi \phi \theta} &\xleq{*} - R_{\phi \phi \theta \phi} \: ,
    \end{split}
    % \begin{split}
    %     R^\phi_{\phi \phi \phi} &= \Gamma^\phi_{\phi \phi,\phi} - \Gamma^\phi_{\phi \phi,\phi} + \Gamma^\phi_{e\phi} \Gamma^e_{\phi \phi} - \Gamma^\phi_{e\phi} \Gamma^e_{\phi \phi} \\
    %         &= 0 \: .
    % \end{split}
\end{align}
\end{subequations}
hvor vi (ved *) har gjort brug af antisymmetriidentiteten, $R_{abcd} = - R_{bacd} = - R_{abdc}$, for Riemanntensoren, \cite[ligning 12--15]{ugeseddel7}.

For nu at beregne Riccitensoren benytter vi, at denne er er givet ved Riemanntensoren,
\begin{align}
    R_{ab} &= R^d_{adb} \: ,
\end{align}
hvorved vi får
\begin{subequations}
\begin{align}
    %%%%%%%%%%%%%%%%%%%%%%%%%%%%%%%%%%%%%%%%%%%%%%%%%%
    % \theta\theta
    %%%%%%%%%%%%%%%%%%%%%%%%%%%%%%%%%%%%%%%%%%%%%%%%%%
    \begin{split}
        R_{\theta\theta} &= R^d_{\theta d \theta} \\
            &= g^{dd} R_{d \theta d \theta} \\
            &= g^{\theta\theta} R_{\theta \theta \theta \theta} + g^{\phi\phi} R_{\phi \theta \phi \theta} \\
            &= \inv{r^2} \cdot 0 + \inv{r^2 \sin^2(\theta)} \cdot r^2 \sin^2(\theta) \\
            &= 1 \: ,
    \end{split} \\
    %%%%%%%%%%%%%%%%%%%%%%%%%%%%%%%%%%%%%%%%%%%%%%%%%%
    % \theta\phi
    %%%%%%%%%%%%%%%%%%%%%%%%%%%%%%%%%%%%%%%%%%%%%%%%%%
    \begin{split}
        R_{\theta\phi} &= R^d_{\theta d \phi} \\
            &= g^{dd} R_{d \theta d \phi} \\
            &= g^{\theta\theta} R_{\theta \theta \theta \phi} + g^{\phi\phi} R_{\phi \theta \phi \phi} \\
            &= \inv{r^2} \cdot 0 + \inv{r^2 \sin^2(\theta)} \cdot 0 \\
            &= 0 \: ,
    \end{split} \\
    %%%%%%%%%%%%%%%%%%%%%%%%%%%%%%%%%%%%%%%%%%%%%%%%%%
    % \phi\theta
    %%%%%%%%%%%%%%%%%%%%%%%%%%%%%%%%%%%%%%%%%%%%%%%%%%
    \begin{split}
        R_{\phi\theta} &= R^d_{\phi d \theta} \\
            &= g^{dd} R_{d \phi d \theta} \\
            &= g^{\theta\theta} R_{\theta \phi \theta \theta} + g^{\phi\phi} R_{\phi \phi \phi \theta} \\
            &= \inv{r^2} \cdot 0 + \inv{r^2 \sin^2(\theta)} \cdot 0 \\
            &= 0 \: , \quad \text{og}
    \end{split} \\
    %%%%%%%%%%%%%%%%%%%%%%%%%%%%%%%%%%%%%%%%%%%%%%%%%%
    % \phi\phi
    %%%%%%%%%%%%%%%%%%%%%%%%%%%%%%%%%%%%%%%%%%%%%%%%%%
    \begin{split}
        R_{\phi\phi} &= R^d_{\phi d \phi} \\
            &= g^{dd} R_{d \phi d \phi} \\
            &= g^{\theta\theta} R_{\theta \phi \theta \phi} + g^{\phi\phi} R_{\phi \phi \phi \phi} \\
            &= \inv{r^2} r^2 \sin^2(\theta) + \inv{r^2 \sin^2(\theta)} \cdot 0 \\
            &= \sin^2(\theta) \: .
    \end{split}
\end{align}
\end{subequations}

Vi kan nu beregne Ricciskalaren ved
\begin{align}
    R &= g^{ab} R_{ab} \: ,
\end{align}
hvorved vi får
\begin{align}
\begin{split}
    R &= g^{aa} R_{aa} \\
        &= g^{\theta\theta} R_{\theta\theta} + g^{\phi\phi} R_{\phi\phi} \\
        &= \inv{r^2} \cdot 1 + \inv{r^2 \sin^2(\theta)} \sin^2(\theta) \\
        &= \frac{2}{r^2} \: ,
\end{split}
\end{align}
da kun $g^{aa} \ne 0$.
\\

Vi får altså de ikkeforsvindende komponenter af Riemanntensoren til
\begin{align}
    R_{\theta\phi\theta\phi} &= r^2 \sin^2(\theta)
        = R_{\phi\theta\phi\theta}
        = - R_{\theta\phi\phi\theta}
        = - R_{\phi\theta\theta\phi} \: ,
\end{align}
Riccitensoren til
\begin{align}
    R_{\theta\theta} &= 1 \: ,
        \quad
    R_{\phi\phi} = \sin^2(\theta) \: ,
        \quad \text{og} \quad
    R_{\theta\phi} = R_{\phi\theta} = 0 \: ,
\end{align}
samt Ricciskalaren til
\begin{align}
    R &= \frac{2}{r^2} \: .
\end{align}



%%%%%%%%%%%%%%%%%%%%%%%%%%%%%%%%%%%%%%%%%%%%%%%%%%%%%%%%%%%%%%%%%%%%%%%%%%%%%%%%%%%%%

\end{document}