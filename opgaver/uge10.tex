\documentclass[../main.tex]{subfiles}

\begin{document}

%%%%%%%%%%%%%%%%%%%%%%%%%%%%%%%%%%%%%%%%%%%%%%%%%%%%%%%%%%%%%%%%%%%%%%%%%%%%%%%%%%%%%

\section{Uge 10 -- Klassiske test af generel relativitetsteori}
\setcounter{section}{10}


%%%%%%%%%%%%%%%%%%%%%%%%%%%%%%%%%%%%%%%%%%%%%%%%%%%%%%%%%%%%%%%%%%%%%%%%%%%%%%%%

\subsection{(S) Opgave 1 -- Keplers lov i Schwarzschildmetrikken}
\setcounter{subsection}{1}
\setcounter{equation}{0}

Udled Keplers lov (relationen mellem et kredsløbs omløbstid og radius) for en cirkulær bane i Schwarzschildmetrikken.

Svar: Som i Newtonsk teori er $\omega^2 = M/r^3$.

Hint: Omløbstiden er lig $2\pi/\omega$, hvor $\omega = \dd \phi/\dd t$ er vinkelfrekvensen, som kan findes fra geodæten $\DD u^r = 0$.

%%%%%%%%%%%%%%%%%%%%%%%%%

\subsubsection*{Besvarelse}

%%%%%%%%%%%%%%%%%%%%%%%%%

\paragraph{Metode 1: Første metode som jeg udregnede i TØ}$ $\\ \vspace{-1em}

Det generelle metriske linjeelement for et ikkeroterende objekt i et graviationelt felt er
\begin{align}
    \dd s^2 &= g_{tt} c^2 \dd t^2 - g_{rr} \dd r^2 - g_{\phi\phi} \left[ \dd \theta^2 + \sin^2(\theta) \dd \phi^2 \right] \: ,
\end{align}
hvor de metriske koefficienter kun er afhængige af det radiære koordinat.

Uden tab af generalitet kan vi kigge på et testobjekt, der bevæger sig i det ækvatoriale plan ($\theta = \pi/2$), hvormed Lagrangefunktionen bliver
\begin{align}
    L &= g_{tt} \pfrac{c\, \dd t}{\dd \tau}^2 - g_{rr} \left( \dif{r}{\tau} \right)^2 - g_{\phi\phi} \left( \dif{\phi}{\tau} \right)^2 \: .
\end{align}
Det første integral i Euler-Lagrangeligningen udledes fra variationsprincippet, $\delta \int_\tau L\, \dd \tau = 0$, hvilket svarer til at tidskoordinatet er konstant
\begin{align}
    g_{tt} \pfrac{c\, \dd t}{\dd \tau} &= k \: ,
\end{align}
hvor $k$ er en arbitrær integrationskonstant.

Euler-Lagrangeligningen svarer til, at det ridære koordinat er
\begin{align}
    - \dif{}{\tau} \left( 2 g_{rr} \dif{r}{\tau} \right) &= g_{tt,r} \pfrac{c\, \dd t}{\dd \tau}^2 - g_{rr,r} \left( \dif{r}{\tau} \right)^2 - g_{\phi\phi,r} \left( \dif{\phi}{\tau} \right)^2 \: ,
\end{align}
hvor $g_{ii,r} = \partial g_{ii} / \partial r$ er den partielle afledede af $g_{ii}$ mht. det radiære koordinat.

For en cirkulær bane er $r = \mathrm{konst}$, hvorfor
\begin{align}
    \dif{r}{\tau} &= 0
        \qquad \text{og} \qquad
    \dif[2]{r}{\tau} = 0 \: .
\end{align}
Benytter vi grænsen for den cirkulære bane på differentialerne fra Euler-Lagrangeligningen fås
\begin{subequations}
\begin{align}
    \Lim{r}{\mathrm{konst}} \left( \frac{\dd \phi / \dd \tau}{c\, \dd t / \dd \tau} \right) &= \sqf{g_{tt,r}}{g_{\phi\phi,r}} \: ,
        \quad \text{og} \\
    \Lim{r}{\mathrm{konst}} \left( \frac{\dd \phi / \dd \tau}{c\, \dd t / \dd \tau} \right) &= \Lim{r}{\mathrm{konst}} \left( \frac{(\dd \phi / \dd t) (\dd t / \dd \tau)}{c\, \dd t / \dd \tau} \right)
        = \inv{c} \dif{\phi}{t} \: ,
\end{align}
\end{subequations}
hvor $\dd t / \dd \tau$ er blevet ''fjernet'' uden problemer, da det er klart at $g_{tt} \dd t / \dd \tau = k$ er endelig og forskellig fra $0$.

Indsætter vi ovenstående ligninger i udtrykket fundet fra Euler-Lagranngeligningen fås
\begin{align}
    \left(\dif{\phi}{t}\right) &= c^2 \frac{g_{tt,r}}{g_{\phi\phi,r}} \: .
\end{align}

Fra Schwarzschildmetrikken ved vi, at de metriske koefficienter er
\begin{subequations}
\begin{align}
    g_{tt} &= 1 - \frac{r_s}{r} \: , \\
    g_{rr} &= \left(1 - \frac{r_s}{r}\right)^{-1} \qquad \text{og} \\
    g_{\phi\phi} &= r^2 \: ,
\end{align}
\end{subequations}
og de radiære afledede er
\begin{subequations}
\begin{align}
    g_{tt,r} &= \pdif{g_{tt}}{r}
        = \pdif{}{r} \left(1 - \frac{r_s}{r}\right)
        = \frac{r_s}{r^2}
    \qquad \text{og} \\
    g_{\phi\phi} &= \pdif{g_{\phi\phi}}{r}
        = \pdif{}{r} \left(r^2 \right)
        = 2r \: .
\end{align}
\end{subequations}
Siden
\begin{align}
    \dif{\phi}{t} &= \omega = \pfrac{\phi}{t_0} = \frac{2\pi}{t_0} \: ,
\end{align}
hvor $t_0$ er omløbstiden, så er
\begin{align}
    \pfrac{2\pi}{t_0}^2 &= \omega^2 = c^2 \frac{g_{tt,r}}{g_{\phi\phi,r}}
        \quad \Rightarrow \quad
    t_0 = \frac{2\pi}{c} \sqf{g_{\phi\phi,r}}{g_{tt,r}} \: ,
\end{align}
og indsætter vi de fundne afledede heri fås
\begin{align} \label{eq:Uge10_Opg1_Methos1_KeplerThirdLawFound}
\begin{split}
    t_0 &= \frac{2\pi}{c} \sqf{2r}{r_s / r^2}
        = \frac{2\pi}{c} \sqf{2r^3}{r_s}
        = \frac{2\pi}{c} \sqf{2r^3}{2GM/c^2}
        = \frac{2\pi}{c} c \sqf{r^3}{GM}
        = 2\pi \sqf{r^3}{G M}
\end{split}
\end{align}
i Schwarszchildmetrikken, da $r_s = 2GM/c^2$ i den Newtonske grænse, \cite[side 2]{ugeseddel9}.

Keplers tredje lov er i Newtonsk mekanik givet som
\begin{align}
    t_0^2 &= \frac{4 \pi^2}{G M} r^3
        \quad \Rightarrow \quad
    t_0 = 2\pi \sqf{r^3}{G M} \: ,
\end{align}
hvilket præcis er den fundre relation fra \cref{eq:Uge10_Opg1_Methos1_KeplerThirdLawFound}.


%%%%%%%%%%%%%%%%%%%%%%%%%

\paragraph{Metode 2: Mere à la Dmitris tankegang}$ $\\ \vspace{-1em}

For cirkulære baner er der ingen ændring i det radiære koordinat, hvorfor
\begin{align}
    \DD u^r &= 0
        \Rightarrow
    0 = \dif{u^r}{s}
        = \Gamma^r_{bc} u^b u^c
        = \Gamma^r_{tt} (u^t)^2 + \Gamma^r_{rr} (u^r)^2 + \Gamma^r_{\phi\phi} (u^\phi)^2 \: .
\end{align}
Yderligere vides det, at $\Gamma^r_{rr} = 0$, da \ldots, hvorfor
\begin{align} \label{eq:Uge10_Opg1_Method2_RelationBetweenGammas}
    \Gamma^r_{tt} (u^t)^2 &= - \Gamma^r_{\phi\phi} (u^\phi)^2 \: .
\end{align}
Vi vil nu først beregne Christoffelsymbolet på venstresiden, altså $\Gamma^r_{tt} (u^t)^2$. Til dette er de relevante metriske koefficienter
\begin{align}
    g_{rr} &= -\left( 1 - \frac{r_s}{r} \right)^{-1}
        \quad \text{og} \quad
    g_{tt} = 1 - \frac{r_s}{r} \: ,
\end{align}
og den relevante afledede er
\begin{align}
    g_{tt,r} &= \frac{r_s}{r^2} \: .
\end{align}
Dermed bliver Christoffelsymbolet
\begin{align}
    \Gamma^r_{tt} (u^t)^2 &= \inv{2} g^{rr} (g_{rt,t} - g_{tt,r} + g_{tr,t})
        = \ldots \: ,
\end{align}
da $g_{ab,c} = g_{ba,c}$, samt $\Gamma_{abc} = (g_{ab,c} - g_{bc,a} + g_{ac,b})/2$ og $g^{aa} = (g_{aa})^{-1}$.

\ldots



%%%%%%%%%%%%%%%%%%%%%%%%%%%%%%%%%%%%%%%%%%%%%%%%%%%%%%%%%%%%%%%%%%%%%%%%%%%%%%%%

\subsection{(M) Opgave 2 -- Perihelion præcession grundet radiær perturbation i næsten cirkulær planetbane}
\setcounter{subsection}{2}
\setcounter{equation}{0}

Betragt en næsten cirkulær bane omkring en stjerne i Newtonsk mekanik og i generel relativitetsteori. Udled ligningen for en lille radiær perturbation i banen, og find dens omløbstid. Relatér til den post-Newtonske perihelion præcession.

%%%%%%%%%%%%%%%%%%%%%%%%%

\subsubsection*{Besvarelse}

\ldots



%%%%%%%%%%%%%%%%%%%%%%%%%%%%%%%%%%%%%%%%%%%%%%%%%%%%%%%%%%%%%%%%%%%%%%%%%%%%%%%%

\subsection{(M) Opgave 3 -- Perihelion præcession grundet ikkesfærisk stjerne}
\setcounter{subsection}{3}
\setcounter{equation}{0}

Betragt i den Newtonske mekanik en planet, som roterer omkring en svagt ikkesfærisk stjerne, sådan at det klassiske Newtonske gavitationelle potential er giver som
\begin{align}
    \phi(r) &= - \frac{M}{r} - \frac{\alpha M}{r^3} \: ,
\end{align}
hvor den lille parameter $\alpha$ beskriver stjernens ikkesfæriskhed (eng: non-sphericity). Beregn i den Newtonske mekanik præcessionen af planetbanens perihelion. For simplicitet kan det antages, at banen er næsten cirkulær.

%%%%%%%%%%%%%%%%%%%%%%%%%

\subsubsection*{Besvarelse}

\ldots



%%%%%%%%%%%%%%%%%%%%%%%%%%%%%%%%%%%%%%%%%%%%%%%%%%%%%%%%%%%%%%%%%%%%%%%%%%%%%%%%

\subsection{(M) Opgave 4 -- Tid er geodæter i synkron referenceramme}
\setcounter{subsection}{4}
\setcounter{equation}{0}

Vis at tidslinjerne (eng: time lines) er geodæter i en synkron referenceramme ($\dd s^2 = \dd \tau^2 + g_{\alpha\beta} \dd x^a \dd x^b$, hvor $\alpha,\beta = 1,2,3$).

%%%%%%%%%%%%%%%%%%%%%%%%%

\subsubsection*{Besvarelse}

\ldots



%%%%%%%%%%%%%%%%%%%%%%%%%%%%%%%%%%%%%%%%%%%%%%%%%%%%%%%%%%%%%%%%%%%%%%%%%%%%%%%%%%%%%

\end{document}