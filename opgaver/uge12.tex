\documentclass[../main.tex]{subfiles}

\begin{document}

%%%%%%%%%%%%%%%%%%%%%%%%%%%%%%%%%%%%%%%%%%%%%%%%%%%%%%%%%%%%%%%%%%%%%%%%%%%%%%%%%%%%%

\section{Uge 12 -- Friedmanunivers (FLRW-univers)}
\setcounter{section}{12}


%%%%%%%%%%%%%%%%%%%%%%%%%%%%%%%%%%%%%%%%%%%%%%%%%%%%%%%%%%%%%%%%%%%%%%%%%%%%%%%%

\subsection{Opgave 1 -- Omskriv ${}^\eta_\eta$-Friedmanligningen til ${}^t_t$-versionen}
\setcounter{subsection}{1}
\setcounter{equation}{0}

Vores ${}^\eta_\eta$-Friedmanligning er skrevet for den $\eta$-afledede, $a' \doteq \dd a/\dd\eta$. Omskriv den (som i Wikeperdiaartiklen\footnote{Muligvis denne artikel: \url{https://en.wikipedia.org/wiki/Friedmann_equations}.}) for den $t$-afledede, $\dt{a} = \dd a/\dd t$, i stedet.

%%%%%%%%%%%%%%%%%%%%%%%%%

\subsubsection*{Besvarelse}

${}^\eta_\eta$-komponenten af Friedmanligningen for det lukkede ($+$) og åbne ($-$) univers er \cite[ligning 15 og 23]{ugeseddel12}
\begin{align}
    \frac{3}{a^4} \left( a^2 \pm a'^2 \right) &= \kappa \epsilon \: .
\end{align}
Vi ved også, at $\dd t = a \dd \eta$, \cite[ligning 11]{ugeseddel12}, hvorfor
\begin{align}
    a' &= \pdif{a}{\eta}
        = \pdif{a}{t}\, \pdif{t}{\eta}
        = \dot{a} a \: ,
\end{align}
hvorfor vi får ${}^t_t$-komponenten af Friedmanligningen for det lukkede ($+$) og åbne ($-$) univers til
\begin{align}
\begin{split}
    \kappa \epsilon &= \frac{3}{a^4} \left( a^2 \pm a'^2 \right) \\
        &= \frac{3}{a^4} \left( a^2 \pm \left[ \dt{a} a \right]^2 \right) \\
        &= \frac{3}{a^4} \left( a^2 \pm a^2 \dt{a}^2 \right) \\
        &= \frac{3 a^2}{a^4} \left( 1 \pm \dt{a}^2 \right) \\
        &= \frac{3}{a^2} \left( 1 \pm \dt{a}^2 \right) \: . \\
    \Rightarrow
    \pfrac{\dt{a}}{a}^2 &= \pm \frac{\kappa \epsilon}{3} \mp \inv{a^2} \: .
\end{split}
\end{align}



%%%%%%%%%%%%%%%%%%%%%%%%%%%%%%%%%%%%%%%%%%%%%%%%%%%%%%%%%%%%%%%%%%%%%%%%%%%%%%%%

\subsection{Opgave 2 -- Riccitensor i Friedmankoordinater for lukket univers}
\setcounter{subsection}{2}
\setcounter{equation}{0}

Beregn manuelt Riccitensoren i Friedmankoordinater for et lukket univers.

%%%%%%%%%%%%%%%%%%%%%%%%%

\subsubsection*{Besvarelse}

For et lukket univers er Friedmanmetrikken
\begin{align}
    \dd s^2 &= a^2 \Big[ \dd \eta^2 - \dd \chi^2 - \sin^2(\chi) \left\{ \dd \theta^2 + \sin^2(\theta)\, \dd \phi^2 \right\} \Big] \: ,
\end{align}
hvorfor
\begin{align}
    g_{\eta\eta} &= a^2 \: , \quad
    g_{\chi\chi} = - a^2 \: , \quad
    g_{\theta\theta} = - a^2 \sin^2(\chi) \: , \quad \text{og} \quad
    g_{\phi\phi} = - a^2 \sin^2(\chi) \sin^2(\theta) \: ,
\end{align}
mens alle andre kominationer af $g_{ab} = 0$, da $\dd s^2 = \sum_{a,b} g_{ab}\, \dd x^a\, \dd x^b$.

Da vi skal beregne Riccitensoren, som består af Riemanntensoren, som består af Christoffelsymboler, da beregner vi først disse, hvilket gøres ved
\begin{align}
    \Gamma_{abc} &= \inv{2} \left[ g_{ab,c} - g_{bc,a} + g_{ca,b} \right] \: .
\end{align}
Christoffelsymbolerne bliver derved
\begin{subequations}
\begin{align}
    %%%%%%%%%%%%%%%%%%%%%%%%%%%%%%%%%%%%%%%%%%%%%%%%%%
    % a = \eta, b = c
    %%%%%%%%%%%%%%%%%%%%%%%%%%%%%%%%%%%%%%%%%%%%%%%%%%
    \begin{split}
        \Gamma^\eta_{\eta\eta} &= g^{\eta\eta} \Gamma_{\eta\eta\eta}
            = \frac{g^{\eta\eta}}{2} \left[ g_{\eta\eta,\eta} - g_{\eta\eta,\eta} + g_{\eta\eta,\eta} \right]
            = \frac{g^{\eta\eta}}{2} g_{\eta\eta,\eta}
            = \inv{2a^2} \pdif{a^2}{\eta} \\
            &= \frac{a'}{a} \: ,
    \end{split} \\
    \begin{split}
        \Gamma^\eta_{\chi\chi} &= g^{\eta\eta} \Gamma_{\eta\chi\chi}
            = \frac{g^{\eta\eta}}{2} \left[ g_{\eta\chi,\chi} - g_{\chi\chi,\eta} + g_{\chi\eta,\chi} \right]
            = \frac{g^{\eta\eta}}{2} \left[ - g_{\chi\chi,\eta} \right]
            = \frac{-1}{2a^2} \pdif{(-a^2)}{\eta} \\
            &= \frac{a'}{a} \: ,
    \end{split} \\
    \begin{split}
        \Gamma^\eta_{\theta\theta} &= g^{\eta\eta} \Gamma_{\eta\theta\theta}
            = \frac{g^{\eta\eta}}{2} \left[ g_{\eta\theta,\theta} - g_{\theta\theta,\eta} + g_{\theta\eta,\theta} \right]
            = \frac{g^{\eta\eta}}{2} \left[ - g_{\theta\theta,\eta} \right]
            = \frac{-1}{2a^2} \pdif{\{-a^2\sin^2(\chi)\}}{\eta} \\
            &= \frac{a'}{a} \sin^2(\chi) \: ,
    \end{split} \\
    \begin{split}
        \Gamma^\eta_{\phi\phi} &= g^{\eta\eta} \Gamma_{\eta\phi\phi}
            = \frac{g^{\eta\eta}}{2} \left[ g_{\eta\phi,\phi} - g_{\phi\phi,\eta} + g_{\phi\eta,\phi} \right]
            = \frac{g^{\eta\eta}}{2} \left[ - g_{\phi\phi,\eta} \right]
            = \frac{-1}{2a^2} \pdif{\{- a^2 \sin^2(\chi)\sin^2(\theta)\}}{\eta} \\
            &= \frac{a'}{a} \sin^2(\chi)\sin^2(\theta) \: ,
    \end{split} \\
    %%%%%%%%%%%%%%%%%%%%%%%%%%%%%%%%%%%%%%%%%%%%%%%%%%
    % a = \eta, b = \eta =\= c
    %%%%%%%%%%%%%%%%%%%%%%%%%%%%%%%%%%%%%%%%%%%%%%%%%%
    \begin{split}
        \Gamma^\eta_{\eta\chi} &= g^{\eta\eta} \Gamma_{\eta\eta\chi}
            = \frac{g^{\eta\eta}}{2} \left[ g_{\eta\eta,\chi} - g_{\eta\chi,\eta} + g_{\chi\eta,\eta} \right]
            = \frac{g^{\eta\eta}}{2} g_{\eta\eta,\chi}
            = \inv{2a^2} \pdif{a^2}{\chi} \\
            &= 0 \: ,
    \end{split} \\
    \begin{split}
        \Gamma^\eta_{\eta\theta} &= g^{\eta\eta} \Gamma_{\eta\eta\theta}
            = \frac{g^{\eta\eta}}{2} \left[ g_{\eta\eta,\theta} - g_{\eta\theta,\eta} + g_{\theta\eta,\eta} \right]
            = \frac{g^{\eta\eta}}{2} g_{\eta\eta,\theta}
            = \inv{2} \pdif{a^2}{\theta} \\
            &= 0 \: ,
    \end{split} \\
    \begin{split}
        \Gamma^\eta_{\eta\phi} &= g^{\eta\eta} \Gamma_{\eta\eta\phi}
            = \frac{g^{\eta\eta}}{2} \left[ g_{\eta\eta,\phi} - g_{\eta\phi,\eta} + g_{\phi\eta,\eta} \right]
            = \frac{g^{\eta\eta}}{2} g_{\eta\eta,\phi}
            = \inv{2} \pdif{a^2}{\phi} \\
            &= 0 \: ,
    \end{split} \\
    %%%%%%%%%%%%%%%%%%%%%%%%%%%%%%%%%%%%%%%%%%%%%%%%%%
    % a = \chi, b = c
    %%%%%%%%%%%%%%%%%%%%%%%%%%%%%%%%%%%%%%%%%%%%%%%%%%
    \begin{split}
        \Gamma^\chi_{\eta\eta} &= g^{\chi\chi} \Gamma_{\chi\eta\eta}
            = \frac{g^{\chi\chi}}{2} \left[ g_{\chi\eta,\eta} - g_{\eta\eta,\chi} + g_{\eta\chi,\eta} \right]
            = \frac{g^{\chi\chi}}{2} \left[ - g_{\eta\eta,\chi} \right]
            = \frac{-1}{2(-a^2)} \pdif{a^2}{\chi} \\
            &= 0 \: ,
    \end{split} \\
    \begin{split}
        \Gamma^\chi_{\chi\chi} &= g^{\chi\chi} \Gamma_{\chi\chi\chi}
            = \frac{g^{\chi\chi}}{2} \left[ g_{\chi\chi,\chi} - g_{\chi\chi,\chi} + g_{\chi\chi,\chi} \right]
            = \frac{g^{\chi\chi}}{2} g_{\chi\chi,\chi}
            = \inv{2(-a^2)} \pdif{(-a^2)}{\chi} \\
            &= 0 \: ,
    \end{split} \\
    \begin{split}
        \Gamma^\chi_{\theta\theta} &= g^{\chi\chi} \Gamma_{\chi\theta\theta}
            = \frac{g^{\chi\chi}}{2} \left[ g_{\chi\theta,\theta} - g_{\theta\theta,\chi} + g_{\theta\chi,\theta} \right]
            = \frac{g^{\chi\chi}}{2} \left[ - g_{\theta\theta,\chi} \right]
            = \frac{-1}{2(-a^2)} \pdif{\{-a^2\sin^2(\chi)\}}{\chi} \\
            &= - \sin(\chi) \cos(\chi) \: ,
    \end{split} \\
    \begin{split}
        \Gamma^\chi_{\phi\phi} &= g^{\chi\chi} \Gamma_{\chi\phi\phi}
            = \frac{g^{\chi\chi}}{2} \left[ g_{\chi\phi,\phi} - g_{\phi\phi,\chi} + g_{\phi\chi,\phi} \right]
            = \frac{g^{\chi\chi}}{2} \left[ - g_{\phi\phi,\chi} \right]
            = \frac{-1}{2(-a^2)} \pdif{\{-a^2\sin^2(\chi)\sin^2(\theta)\}}{\chi} \\
            &= - \sin(\chi) \cos(\chi) \sin^2(\theta) \: ,
    \end{split} \\
    %%%%%%%%%%%%%%%%%%%%%%%%%%%%%%%%%%%%%%%%%%%%%%%%%%
    % a = \chi, b = \chi =\= c
    %%%%%%%%%%%%%%%%%%%%%%%%%%%%%%%%%%%%%%%%%%%%%%%%%%
    \begin{split}
        \Gamma^\chi_{\chi\eta} &= g^{\chi\chi} \Gamma_{\chi\chi\eta}
            = \frac{g^{\chi\chi}}{2} \left[ g_{\chi\chi,\eta} - g_{\chi\eta,\chi} + g_{\eta\chi,\chi} \right]
            = \frac{g^{\chi\chi}}{2} g_{\chi\chi,\eta}
            = \inv{2(-a^2)} \pdif{(-a^2)}{\eta} \\
            = \frac{a'}{a} \: ,
    \end{split} \\
    \begin{split}
        \Gamma^\chi_{\chi\theta} &= g^{\chi\chi} \Gamma_{\chi\chi\theta}
            = \frac{g^{\chi\chi}}{2} \left[ g_{\chi\chi,\theta} - g_{\chi\theta,\chi} + g_{\theta\chi,\chi} \right]
            = \frac{g^{\chi\chi}}{2} g_{\chi\chi,\theta}
            = \inv{2(-a^2)} \pdif{(-a^2)}{\theta} \\
            &= 0 \: ,
    \end{split} \\
    \begin{split}
        \Gamma^\chi_{\chi\phi} &= g^{\chi\chi} \Gamma_{\chi\chi\phi}
            = \frac{g^{\chi\chi}}{2} \left[ g_{\chi\chi,\phi} - g_{\chi\phi,\chi} + g_{\phi\chi,\chi} \right]
            = \frac{g^{\chi\chi}}{2} g_{\chi\chi,\phi}
            = \inv{2(-a^2)} \pdif{(-a^2)}{\phi} \\
            &= 0 \: ,
    \end{split} \\
    %%%%%%%%%%%%%%%%%%%%%%%%%%%%%%%%%%%%%%%%%%%%%%%%%%
    % a = \theta, b = c
    %%%%%%%%%%%%%%%%%%%%%%%%%%%%%%%%%%%%%%%%%%%%%%%%%%
    \begin{split}
        \Gamma^\theta_{\eta\eta} &= g^{\theta\theta} \Gamma_{\theta\eta\eta}
            = \frac{g^{\theta\theta}}{2} \left[ g_{\theta\eta,\eta} - g_{\eta\eta,\theta} + g_{\eta\theta,\eta} \right]
            = \frac{g^{\theta\theta}}{2} \left[ - g_{\eta\eta,\theta} \right]
            = \frac{-1}{2[-a^2\sin^2(\chi)]} \pdif{a^2}{\theta} \\
            &= 0 \: ,
    \end{split} \\
    \begin{split}
        \Gamma^\theta_{\chi\chi} &= g^{\theta\theta} \Gamma_{\theta\chi\chi}
            = \frac{g^{\theta\theta}}{2} \left[ g_{\theta\chi,\chi} - g_{\chi\chi,\theta} + g_{\chi\theta,\chi} \right]
            = \frac{g^{\theta\theta}}{2} \left[ - g_{\chi\chi,\theta} \right]
            = \frac{-1}{2[-a^2\sin^2(\chi)]} \pdif{(-a^2)}{\theta} \\
            &= 0 \: ,
    \end{split} \\
    \begin{split}
        \Gamma^\theta_{\theta\theta} &= g^{\theta\theta} \Gamma_{\theta\theta\theta}
            = \frac{g^{\theta\theta}}{2} \left[ g_{\theta\theta,\theta} - g_{\theta\theta,\theta} + g_{\theta\theta,\theta} \right]
            = \frac{g^{\theta\theta}}{2} g_{\theta\theta,\theta}
            = \inv{2[-a^2\sin^2(\chi)]} \pdif{\{-a^2 \sin^2(\chi)\}}{\theta} \\
            &= 0 \: ,
    \end{split} \\
    \begin{split}
        \Gamma^\theta_{\phi\phi} &= g^{\theta\theta} \Gamma_{\theta\phi\phi}
            = \frac{g^{\theta\theta}}{2} \left[ g_{\theta\phi,\phi} - g_{\phi\phi,\theta} + g_{\phi\theta,\phi} \right]
            = \frac{g^{\theta\theta}}{2} \left[ - g_{\phi\phi,\theta} \right]
            = \frac{-1}{2[-a^2\sin^2(\chi)]} \pdif{\{-a^2 \sin^2(\chi)\sin^2(\theta)\}}{\theta} \\
            &= - \sin(\theta) \cos(\theta) \: ,
    \end{split} \\
    %%%%%%%%%%%%%%%%%%%%%%%%%%%%%%%%%%%%%%%%%%%%%%%%%%
    % a = \theta, b = \theta =\= c
    %%%%%%%%%%%%%%%%%%%%%%%%%%%%%%%%%%%%%%%%%%%%%%%%%%
    \begin{split}
        \Gamma^\theta_{\theta\eta} &= g^{\theta\theta} \Gamma_{\theta\theta\eta}
            = \frac{g^{\theta\theta}}{2} \left[ g_{\theta\theta,\eta} - g_{\theta\eta,\theta} + g_{\eta\theta,\theta} \right]
            = \frac{g^{\theta\theta}}{2} g_{\theta\theta,\eta}
            = \inv{2[-a^2\sin^2(\chi)]} \pdif{\{-a^2 \sin^2(\chi)\}}{\eta} \\
            &= \frac{a'}{a} \: ,
    \end{split} \\
    \begin{split}
        \Gamma^\theta_{\theta\chi} &= g^{\theta\theta} \Gamma_{\theta\theta\chi}
            = \frac{g^{\theta\theta}}{2} \left[ g_{\theta\theta,\chi} - g_{\theta\chi,\theta} + g_{\chi\theta,\theta} \right]
            = \frac{g^{\theta\theta}}{2} g_{\theta\theta,\chi}
            = \inv{2[-a^2\sin^2(\chi)]} \pdif{\{-a^2 \sin^2(\chi)\}}{\chi} \\
            &= \frac{\cos(\chi)}{\sin(\chi)}
            = \cot(\chi) \: ,
    \end{split} \\
    \begin{split}
        \Gamma^\theta_{\theta\phi} &= g^{\theta\theta} \Gamma_{\theta\theta\phi}
            = \frac{g^{\theta\theta}}{2} \left[ g_{\theta\theta,\phi} - g_{\theta\phi,\theta} + g_{\phi\theta,\theta} \right]
            = \frac{g^{\theta\theta}}{2} g_{\theta\theta,\phi}
            = \inv{2[-a^2\sin^2(\chi)]} \pdif{\{-a^2 \sin^2(\chi)\}}{\phi} \\
            &= 0 \: ,
    \end{split} \\
    %%%%%%%%%%%%%%%%%%%%%%%%%%%%%%%%%%%%%%%%%%%%%%%%%%
    % a = \phi, b = c
    %%%%%%%%%%%%%%%%%%%%%%%%%%%%%%%%%%%%%%%%%%%%%%%%%%
    \begin{split}
        \Gamma^\phi_{\eta\eta} &= g^{\phi\phi} \Gamma_{\phi\eta\eta}
            = \frac{g^{\phi\phi}}{2} \left[ g_{\phi\eta,\eta} - g_{\eta\eta,\phi} + g_{\eta\phi,\eta} \right]
            = \frac{g^{\phi\phi}}{2} \left[ - g_{\eta\eta,\phi} \right]
            = \frac{-1}{2[-a^2 \sin^2(\chi) \sin^2(\theta)]} \pdif{a^2}{\phi} \\
            &= 0 \: ,
    \end{split} \\
    \begin{split}
        \Gamma^\phi_{\chi\chi} &= g^{\phi\phi} \Gamma_{\phi\chi\chi}
            = \frac{g^{\phi\phi}}{2} \left[ g_{\phi\chi,\chi} - g_{\chi\chi,\phi} + g_{\chi\phi,\chi} \right]
            = \frac{g^{\phi\phi}}{2} \left[ - g_{\chi\chi,\phi} \right]
            = \frac{g^{\phi\phi}}{2} \pdif{(-a^2)}{\phi} \\
            &= 0 \: ,
    \end{split} \\
    \begin{split}
        \Gamma^\phi_{\theta\theta} &= g^{\phi\phi} \Gamma_{\phi\theta\theta}
            = \frac{g^{\phi\phi}}{2} \left[ g_{\phi\theta,\theta} - g_{\theta\theta,\phi} + g_{\theta\phi,\theta} \right]
            = \frac{g^{\phi\phi}}{2} \left[ - g_{\theta\theta,\phi} \right]
            = \frac{g^{\phi\phi}}{2} \pdif{\{-a^2 \sin^2(\chi)\}}{\phi} \\
            &= 0 \: ,
    \end{split} \\
    \begin{split}
        \Gamma^\phi_{\phi\phi} &= g^{\phi\phi} \Gamma_{\phi\phi\phi}
            = \frac{g^{\phi\phi}}{2} \left[ g_{\phi\phi,\phi} - g_{\phi\phi,\phi} + g_{\phi\phi,\phi} \right]
            = \frac{g^{\phi\phi}}{2} g_{\eta\eta,\phi}
            = \frac{g^{\phi\phi}}{2} \pdif{\{-a^2 \sin^2(\chi) \sin^2(\theta)\}}{\phi} \\
            &= 0 \: ,
    \end{split} \\
    %%%%%%%%%%%%%%%%%%%%%%%%%%%%%%%%%%%%%%%%%%%%%%%%%%
    % a = \phi, b = \phi =\= c
    %%%%%%%%%%%%%%%%%%%%%%%%%%%%%%%%%%%%%%%%%%%%%%%%%%
    \begin{split}
        \Gamma^\phi_{\phi\eta} &= g^{\phi\phi} \Gamma_{\phi\phi\eta}
            = \frac{g^{\phi\phi}}{2} \left[ g_{\phi\phi,\eta} - g_{\phi\eta,\phi} + g_{\eta\phi,\phi} \right]
            = \frac{g^{\phi\phi}}{2} g_{\phi\phi,\eta}
            = \frac{g^{\phi\phi}}{2} \pdif{\{-a^2 \sin^2(\chi) \sin^2(\theta)\}}{\eta} \\
            &= \frac{a'}{a} \: ,
    \end{split} \\
    \begin{split}
        \Gamma^\phi_{\phi\chi} &= g^{\phi\phi} \Gamma_{\phi\phi\chi}
            = \frac{g^{\phi\phi}}{2} \left[ g_{\phi\phi,\chi} - g_{\phi\chi,\phi} + g_{\chi\phi,\phi} \right]
            = \frac{g^{\phi\phi}}{2} g_{\phi\phi,\chi}
            = \frac{g^{\phi\phi}}{2} \pdif{\{-a^2 \sin^2(\chi) \sin^2(\theta)\}}{\chi} \\
            &= \frac{\cos(\chi)}{\sin(\chi)}
            = \cot(\chi) \: ,
    \end{split}
        \tag{\theparentequation æ} \\
    \begin{split}
        \Gamma^\phi_{\phi\theta} &= g^{\phi\phi} \Gamma_{\phi\phi\theta}
            = \frac{g^{\phi\phi}}{2} \left[ g_{\phi\phi,\theta} - g_{\phi\theta,\phi} + g_{\theta\phi,\phi} \right]
            = \frac{g^{\phi\phi}}{2} g_{\phi\phi,\theta}
            = \frac{g^{\phi\phi}}{2} \pdif{\{-a^2 \sin^2(\chi) \sin^2(\theta)\}}{\theta} \\
            &= \frac{\cos(\theta)}{\sin(\theta)}
            = \cot(\theta) \: ,
    \end{split}
        \tag{\theparentequation ø}
\end{align}
\end{subequations}
og Christoffelsymbolerne med alle forskellige indeks ($a \ne b \ne c \ne a$) giver $0$, da kun $g_{aa} \ne 0$.
Yderligere vides det, at $\Gamma_{abc} = \Gamma_{acb}$, hvilket giver de resterende Christoffelsymboler.

Vi benytter nu at Riccitensoren er givet ved Riemanntensoren, som igen er givet ved Christoffelsymbolerne, som
\begin{align}
    R_{ab} &= R^d_{adb}
        = \Gamma^d_{ab,d} - \Gamma^d_{ad,b} + \Gamma^d_{ed} \Gamma^e_{ab} - \Gamma^d_{eb} \Gamma^e_{ad} \: ,
\end{align}
hvorved vi får\\
\begin{subequations}
\begin{align}
    %%%%%%%%%%%%%%%%%%%%%%%%%%%%%%%%%%%%%%%%%%%%%%%%%%
    % \eta\eta
    %%%%%%%%%%%%%%%%%%%%%%%%%%%%%%%%%%%%%%%%%%%%%%%%%%
    \begin{split}
        R^\eta_\eta %&= g^{\eta \eta} R_{\eta \eta}  \\
        %&= g^{\eta \eta} R^d_{\eta d \eta} \\
        &= g^{\eta \eta} \left(\Gamma^d_{\eta\eta,d} - \Gamma^d_{\eta d,\eta} + \Gamma^d_{ed} \Gamma^e_{\eta\eta} - \Gamma^d_{e\eta} \Gamma^e_{\eta d} \right) \\
        &= g^{\eta \eta} \bigg(
            \left[ \Gamma^\eta_{\eta\eta,\eta} + \Gamma^\chi_{\eta\eta,\chi} + \Gamma^\theta_{\eta\eta,\theta} + \Gamma^\phi_{\eta\eta,\phi} \right] \\
            &\qquad\qquad - \left[ \Gamma^\eta_{\eta \eta,\eta} + \Gamma^\chi_{\eta \chi,\eta} + \Gamma^\theta_{\eta \theta,\eta} + \Gamma^\phi_{\eta \phi,\eta} \right] \\
            &\qquad\qquad + \Big[ \left\{ \Gamma^\eta_{\eta\eta} \Gamma^\eta_{\eta\eta} + \Gamma^\chi_{\eta\chi} \Gamma^\eta_{\eta\eta} + \Gamma^\theta_{\eta\theta} \Gamma^\eta_{\eta\eta} + \Gamma^\phi_{\eta\phi} \Gamma^\eta_{\eta\eta} \right\} \\
                &\qquad\qquad\qquad + \left\{ \Gamma^\eta_{\chi\eta} \Gamma^\chi_{\eta\eta} + \Gamma^\chi_{\chi\chi} \Gamma^\chi_{\eta\eta} + \Gamma^\theta_{\chi\theta} \Gamma^\chi_{\eta\eta} + \Gamma^\phi_{\chi\phi} \Gamma^\chi_{\eta\eta} \right\} \\
                &\qquad\qquad\qquad + \left\{ \Gamma^\eta_{\theta\eta} \Gamma^\theta_{\eta\eta} + \Gamma^\chi_{\theta\chi} \Gamma^\theta_{\eta\eta} + \Gamma^\theta_{\theta\theta} \Gamma^\theta_{\eta\eta} + \Gamma^\phi_{\theta\phi} \Gamma^\theta_{\eta\eta} \right\} \\
                &\qquad\qquad\qquad + \left\{ \Gamma^\eta_{\phi\eta} \Gamma^\phi_{\eta\eta} + \Gamma^\chi_{\phi\chi} \Gamma^\phi_{\eta\eta} + \Gamma^\theta_{\phi\theta} \Gamma^\phi_{\eta\eta} + \Gamma^\phi_{\phi\phi} \Gamma^\phi_{\eta\eta} \right\}
            \Big] \\
            &\qquad\qquad - \Big[ \left\{ \Gamma^\eta_{\eta\eta} \Gamma^\eta_{\eta \eta} + \Gamma^\chi_{\eta\eta} \Gamma^\eta_{\eta \chi} + \Gamma^\theta_{\eta\eta} \Gamma^\eta_{\eta \theta} + \Gamma^\phi_{\eta\eta} \Gamma^\eta_{\eta \phi} \right\} \\
                &\qquad\qquad\qquad + \left\{ \Gamma^\eta_{\chi\eta} \Gamma^\chi_{\eta \eta} + \Gamma^\chi_{\chi\eta} \Gamma^\chi_{\eta \chi} + \Gamma^\theta_{\chi\eta} \Gamma^\chi_{\eta \theta} + \Gamma^\phi_{\chi\eta} \Gamma^\chi_{\eta \phi} \right\} \\
                &\qquad\qquad\qquad + \left\{ \Gamma^\eta_{\theta\eta} \Gamma^\theta_{\eta \eta} + \Gamma^\chi_{\theta\eta} \Gamma^\theta_{\eta \chi} + \Gamma^\theta_{\theta\eta} \Gamma^\theta_{\eta \theta} + \Gamma^\phi_{\theta\eta} \Gamma^\theta_{\eta \phi} \right\} \\
                &\qquad\qquad\qquad + \left\{ \Gamma^\eta_{\phi\eta} \Gamma^\phi_{\eta \eta} + \Gamma^\chi_{\phi\eta} \Gamma^\phi_{\eta \chi} + \Gamma^\theta_{\phi\eta} \Gamma^\phi_{\eta \theta} + \Gamma^\phi_{\phi\eta} \Gamma^\phi_{\eta \phi} \right\}
            \Big]
            \bigg) \\
        &= g^{\eta \eta} \bigg( \Gamma^\eta_{\eta\eta,\eta} - \Gamma^\eta_{\eta \eta,\eta} - \Gamma^\chi_{\eta \chi,\eta} - \Gamma^\theta_{\eta \theta,\eta} - \Gamma^\phi_{\eta \phi,\eta} \\
            &\qquad\qquad + \Big[ \Gamma^\eta_{\eta\eta} \Gamma^\eta_{\eta\eta} + \Gamma^\chi_{\eta\chi} \Gamma^\eta_{\eta\eta} + \Gamma^\theta_{\eta\theta} \Gamma^\eta_{\eta\eta} + \Gamma^\phi_{\eta\phi} \Gamma^\eta_{\eta\eta} \Big] \\
            &\qquad\qquad - \Big[ \Gamma^\eta_{\eta\eta} \Gamma^\eta_{\eta \eta} + \Gamma^\chi_{\chi\eta} \Gamma^\chi_{\eta \chi} + \Gamma^\theta_{\theta\eta} \Gamma^\theta_{\eta \theta} + \Gamma^\phi_{\phi\eta} \Gamma^\phi_{\eta \phi} \Big]
            \bigg) \\
        &= g^{\eta \eta} \bigg( - \Gamma^\chi_{\eta \chi,\eta} - \Gamma^\theta_{\eta \theta,\eta} - \Gamma^\phi_{\eta \phi,\eta} + \Big[ \Gamma^\chi_{\eta\chi} + \Gamma^\theta_{\eta\theta} + \Gamma^\phi_{\eta\phi} \Big] \Gamma^\eta_{\eta\eta} \\
            &\qquad\qquad - \left( \Gamma^\chi_{\eta\chi} \right)^2 - \left( \Gamma^\theta_{\eta\theta} \right)^2 - \left( \Gamma^\phi_{\eta \phi} \right)^2
            \bigg) \\
        &= \inv{a^2} \left\{ - 3\pfrac{a'}{a}_{,\eta} + 3\pfrac{a'}{a}^2  - 3\pfrac{a'}{a}^2
            \right\} \\
            % &= - \frac{3}{a^2} \left( \frac{a''}{a} - \frac{a'^2}{a^2} \right) \\
            &= \frac{3}{a^4} \left( a'^2 - aa'' \right) \: ,
    \end{split} \\
    %%%%%%%%%%%%%%%%%%%%%%%%%%%%%%%%%%%%%%%%%%%%%%%%%%
    % \chi\chi
    %%%%%%%%%%%%%%%%%%%%%%%%%%%%%%%%%%%%%%%%%%%%%%%%%%
    \begin{split}
        R^\chi_\chi &= g^{\chi \chi} R_{\chi \chi} 
        = g^{\chi \chi} R^d_{\chi d \chi}
        = g^{\chi \chi} \left(\Gamma^d_{\chi\chi,d} - \Gamma^d_{\chi d,\chi} + \Gamma^d_{ed} \Gamma^e_{\chi\chi} - \Gamma^d_{e\chi} \Gamma^e_{\chi d} \right) \\
        &= \ldots \text{ beregninger som før, dog udeladt grundet tid } \ldots \\
        &= - \inv{a^4} \left( 2a^2 + a'^2 + aa'' \right) \: ,
    \end{split} \\
    %%%%%%%%%%%%%%%%%%%%%%%%%%%%%%%%%%%%%%%%%%%%%%%%%%
    % \theta\theta
    %%%%%%%%%%%%%%%%%%%%%%%%%%%%%%%%%%%%%%%%%%%%%%%%%%
    \begin{split}
        R^\theta_\theta &= g^{\theta \theta} R_{\theta \theta} 
        = g^{\theta \theta} R^d_{\theta d \theta}
        = g^{\theta \theta} \left(\Gamma^d_{\theta\theta,d} - \Gamma^d_{\theta d,\theta} + \Gamma^d_{ed} \Gamma^e_{\theta\theta} - \Gamma^d_{e\theta} \Gamma^e_{\theta d} \right) \\
        &= \ldots \text{ beregninger som før, dog udeladt grundet tid } \ldots \\
        &= - \inv{a^4} \left( 2a^2 + a'^2 + aa'' \right) \: , \quad \text{og}
    \end{split} \\
    %%%%%%%%%%%%%%%%%%%%%%%%%%%%%%%%%%%%%%%%%%%%%%%%%%
    % \phi\phi
    %%%%%%%%%%%%%%%%%%%%%%%%%%%%%%%%%%%%%%%%%%%%%%%%%%
    \begin{split}
        R^\phi_\phi &= g^{\phi \phi} R_{\phi \phi} 
        = g^{\phi \phi} R^d_{\phi d \phi}
        = g^{\phi \phi} \left(\Gamma^d_{\phi\phi,d} - \Gamma^d_{\phi d,\phi} + \Gamma^d_{ed} \Gamma^e_{\phi\phi} - \Gamma^d_{e\phi} \Gamma^e_{\phi d} \right) \\
        &= \ldots \text{ beregninger som før, dog udeladt grundet tid } \ldots \\
        &= - \inv{a^4} \left( 2a^2 + a'^2 + aa'' \right) \: .
    \end{split}
\end{align}
\end{subequations}

Vi kan nu beregne Ricciskalaren ved
\begin{align}
    R &= g^{ab} R_{ab} \: ,
\end{align}
hvorved vi får
\begin{align}
\begin{split}
    R &= g^{aa} R_{aa} \\
        &= g^{\eta\eta} R_{\eta\eta} + g^{\chi\chi} R_{\chi\chi} + g^{\theta\theta} R_{\theta\theta} + g^{\phi\phi} R_{\phi\phi} \\
        &= R^\eta_\eta + R^\chi_\chi + R^\theta_\theta + R^\phi_\phi \\
        &= \frac{3}{a^4} \left( a'^2 - aa'' \right) - \frac{3}{a^4} \left( 2a^2 + a'^2 + aa'' \right) \\
        &= - \frac{6}{a^3} \left( a + a'' \right) \: ,
\end{split}
\end{align}
da kun $g^{aa} \ne 0$.

Dette er præcis de samme værdier som i ligning 12 på ugeseddel 12, \cite[ligning 12]{ugeseddel12}.



%%%%%%%%%%%%%%%%%%%%%%%%%%%%%%%%%%%%%%%%%%%%%%%%%%%%%%%%%%%%%%%%%%%%%%%%%%%%%%%%

\subsection{Opgave 3 -- Energibevarelse fra feltligninger}
\setcounter{subsection}{3}
\setcounter{equation}{0}

Argumentér for at energibevarelsensligningen,
\begin{align}
    (\epsilon a^3)' + p(a^3)' &= 0 \: ,
\end{align}
følger fra feltligningerne
\begin{align}
    \frac{3}{a^4} \left( a^2 + a'^2 \right) &= \kappa \epsilon \: , \quad \text{og} \label{eq:Uge12_Opg3_FieldEquation1} \\
    \inv{a^4} \left( a^2 + 2aa'' - a'^2 \right) &= - \kappa p \: . \label{eq:Uge12_Opg3_FieldEquation2}
\end{align}

%%%%%%%%%%%%%%%%%%%%%%%%%

\subsubsection*{Besvarelse}

Vi betragter først den første feltligning, \cref{eq:Uge12_Opg3_FieldEquation1}, hvori vi isolerer $\epsilon a^3$
\begin{align}
    \kappa \epsilon &= \frac{3}{a^4} \left( a^2 + a'^2 \right)
        \quad \Rightarrow \quad
    \epsilon a^3 = \frac{3}{a \kappa} \left( a^2 + a'^2 \right) \: ,
\end{align}
hvorefter vi nu differentierer dette med hensyn til $\eta$
\begin{align} \label{eq:Uge12_Opg3_FirstPart}
\begin{split}
    (\epsilon a^3)' &= \pdif{}{\eta} \left[ \frac{3}{a \kappa} \left( a^2 + a'^2 \right) \right] \\
        &= \frac{3}{\kappa} \pdif{}{\eta} \left( a + \frac{a'^2}{a} \right) \\
        &= \frac{3}{\kappa} \left( a' + \frac{2a'a''}{a} - \frac{a'^2}{a^2} a' \right) \\
        &= \frac{3 a'}{\kappa} + \frac{6a'a''}{a \kappa} - \frac{3 a'^3}{a^2 \kappa} \: .
\end{split}
\end{align}

Betragter vi nu den anden feltligning, \cref{eq:Uge12_Opg3_FieldEquation2}, hvori vi isolerer $-p$, så får vi
\begin{align}
    - \kappa p &= \inv{a^4} \left( a^2 + 2aa'' - a'^2 \right)
        \quad \Rightarrow \quad
    - p = \inv{a^4 \kappa} \left( a^2 + 2aa'' - a'^2 \right) \: ,
\end{align}
hvorefter vi nu ganger dette på $(a^3)'$, altså ganget på $a^3$ differentieret med hensyn til $\eta$,
\begin{align} \label{eq:Uge12_Opg3_SecondPart}
\begin{split}
    - p (a^3)' &= \inv{a^4 \kappa} \left( a^2 + 2aa'' - a'^2 \right) \pdif{a^3}{\eta} \\
        &= \inv{a^4 \kappa} \left( a^2 + 2aa'' - a'^2 \right) 3a^2 a' \\
        &= \frac{3 a'}{\kappa} + \frac{6a'a''}{a \kappa} - \frac{3 a'^3}{a^2 \kappa} \: .
\end{split}
\end{align}

Det kan nu ses, at vi kan sætte \cref{eq:Uge12_Opg3_FirstPart} lig \cref{eq:Uge12_Opg3_SecondPart}, hvorved vi får
\begin{align}
\begin{split}
    (\epsilon a^3)' &= \frac{3 a'}{\kappa} + \frac{6a'a''}{a \kappa} - \frac{3 a'^3}{a^2 \kappa} = - p (a^3)' \\
    \Rightarrow 0 &= (\epsilon a^3)' + p (a^3)' \: ,
\end{split}
\end{align}
hvilket er energibevarelsesligningen, som vi skulle vise.



%%%%%%%%%%%%%%%%%%%%%%%%%%%%%%%%%%%%%%%%%%%%%%%%%%%%%%%%%%%%%%%%%%%%%%%%%%%%%%%%

\subsection{(S) Opgave 4 -- Tidsudvikling af masse- og strålingsdominerede Euklidiske universer}
\setcounter{subsection}{4}
\setcounter{equation}{0}

Betragt et fladt (Euklidisk) og isotopt univers med metrikken
\begin{align}
    \dd s^2 &= \dd t^2 - a^2(t) \left( \dd x^2 + \dd y^2 + \dd z^2 \right) \: ,
\end{align}
og undersøg dets tidslige udvikling for masse- og strålingsdominerede universer.

Hints:
\begin{enumerate}
    \item Beregn Christoffelsymbolerne
        \begin{align}
            \Gamma^x_{tx} &= \Gamma^y_{ty} = \Gamma^z_{tz} = \frac{\dt{a}}{a} \: ,
                \quad \text{og} \quad
            \Gamma^t_{xx} = \Gamma^t_{yy} = \Gamma^t_{zz} = a \dt{a} \: .
        \end{align}
    \item Beregn Riccitensoren og Ricciskalaren
        \begin{align}
            R^t_t &= - 3 \frac{\ddt{a}}{a} \: ,
                \quad \text{og} \quad
            R^x_x = R^y_y = R^z_z = - \frac{\ddt{a}}{a} - 2 \frac{\dt{a}^2}{a^2} \: .
        \end{align}
    \item Nedskriv ${}^t_t$-komponenten af Einsteinligningen med en perfekt væske
        \begin{align}
            3 \frac{\dt{a}^2}{a^2} &= \kappa \epsilon \: .
        \end{align}
    \item Nedskriv energibevarelsesligningen
        \begin{align}
            \frac{\dd V}{V} &= - \frac{\dd \epsilon}{(\epsilon + p)}
                \quad \Rightarrow \quad
            3 \ln(a) = - \int \frac{\dd \epsilon}{(\epsilon + p)} \: .
        \end{align}
    \item Integrer ligningerne for et massedomineret univers ($p = 0$, $\epsilon = \mu$)
        \begin{align}
            \mu a^3 &= \text{konst} \: ,
                \quad \text{og} \quad
            a \propto t^{2/3} \: .
        \end{align}
    \item Integrer ligningerne for et strålingsdomineret univers ($p = \epsilon / 3$)
        \begin{align}
            \epsilon a^4 &= \text{konst} \: ,
                \quad \text{og} \quad
            a \propto t^{1/2} \: .
        \end{align}
\end{enumerate}

%%%%%%%%%%%%%%%%%%%%%%%%%

\subsubsection*{Besvarelse}

Det flade (Euklidiske) og isotope univers har metrikken
\begin{align}
    \dd s^2 &= \dd t^2 - a^2(t) \left( \dd x^2 + \dd y^2 + \dd z^2 \right) \: ,
\end{align}
hvorfor
\begin{align}
    g_{tt} &= 1 \: , \quad \text{og} \quad
    g_{xx} = g_{yy} = g_{zz} = - a^2(t) \: ,
\end{align}
mens alle andre kominationer af $g_{ab} = 0$, da $\dd s^2 = \sum_{a,b} g_{ab}\, \dd x^a\, \dd x^b$.

Først beregner vi Christoffelsymbolerne, hvilket gøres ved
\begin{align}
    \Gamma_{abc} &= \inv{2} \left[ g_{ab,c} - g_{bc,a} + g_{ca,b} \right] \: .
\end{align}
Christoffelsymbolerne bliver derved
\begin{subequations}
\begin{align}
    %%%%%%%%%%%%%%%%%%%%%%%%%%%%%%%%%%%%%%%%%%%%%%%%%%
    % a = t, b = t, c = t
    %%%%%%%%%%%%%%%%%%%%%%%%%%%%%%%%%%%%%%%%%%%%%%%%%%
    \begin{split}
        \Gamma^t_{tt} &= g^{tt} \Gamma_{ttt}
            = \frac{g^{tt}}{2} \left[ g_{tt,t} - g_{tt,t} + g_{tt,t} \right]
            = \frac{g^{tt}}{2} g_{tt,t}
            = \inv{2 \cdot 1} \pdif{1}{t} \\
            &= 0 \: ,
    \end{split} \\
    %%%%%%%%%%%%%%%%%%%%%%%%%%%%%%%%%%%%%%%%%%%%%%%%%%
    % a = t, b = t, c = r
    %%%%%%%%%%%%%%%%%%%%%%%%%%%%%%%%%%%%%%%%%%%%%%%%%%
    \begin{split}
        \Gamma^t_{tr} &= g^{tt} \Gamma_{ttr}
            = \frac{g^{tt}}{2} \left[ g_{tt,r} - g_{tr,t} + g_{rt,t} \right]
            = \frac{g^{tt}}{2} g_{tt,r}
            = \inv{2 \cdot 1} \pdif{1}{r} \\
            &= 0 \: ,
    \end{split} \\
    %%%%%%%%%%%%%%%%%%%%%%%%%%%%%%%%%%%%%%%%%%%%%%%%%%
    % a = t, b = r, c = r
    %%%%%%%%%%%%%%%%%%%%%%%%%%%%%%%%%%%%%%%%%%%%%%%%%%
    \begin{split}
        \Gamma^t_{rr} &= g^{tt} \Gamma_{trr}
            = \frac{g^{tt}}{2} \left[ g_{tr,r} - g_{rr,t} + g_{rt,r} \right]
            = - \frac{g^{tt}}{2} g_{rr,t}
            = - \inv{2 \cdot 1} \pdif{(- a^2)}{t} \\
            &= a \dt{a} \: ,
    \end{split} \\
    %%%%%%%%%%%%%%%%%%%%%%%%%%%%%%%%%%%%%%%%%%%%%%%%%%
    % a = r, b = r, c = r
    %%%%%%%%%%%%%%%%%%%%%%%%%%%%%%%%%%%%%%%%%%%%%%%%%%
    \begin{split}
        \Gamma^r_{rr} &= g^{rr} \Gamma_{rrr}
            = \frac{g^{rr}}{2} \left[ g_{rr,r} - g_{rr,r} + g_{rr,r} \right]
            = \frac{g^{rr}}{2} g_{rr,r}
            = \inv{2 (- a^2)} \pdif{(- a^2)}{r} \\
            &= 0 \: ,
    \end{split} \\
    %%%%%%%%%%%%%%%%%%%%%%%%%%%%%%%%%%%%%%%%%%%%%%%%%%
    % a = r, b = t, c = r
    %%%%%%%%%%%%%%%%%%%%%%%%%%%%%%%%%%%%%%%%%%%%%%%%%%
    \begin{split}
        \Gamma^r_{tr} &= g^{rr} \Gamma_{rtr}
            = \frac{g^{rr}}{2} \left[ g_{rt,r} - g_{tr,r} + g_{rr,t} \right]
            = \frac{g^{rr}}{2} g_{rr,t}
            = \inv{2 (- a^2)} \pdif{(- a^2)}{t} \\
            &= \frac{\dt{a}}{a} \: ,
    \end{split} \\
    %%%%%%%%%%%%%%%%%%%%%%%%%%%%%%%%%%%%%%%%%%%%%%%%%%
    % a = r, b = t, c = t
    %%%%%%%%%%%%%%%%%%%%%%%%%%%%%%%%%%%%%%%%%%%%%%%%%%
    \begin{split}
        \Gamma^r_{tt} &= g^{rr} \Gamma_{rtt}
            = \frac{g^{rr}}{2} \left[ g_{rt,t} - g_{tt,r} + g_{tr,t} \right]
            = - \frac{g^{rr}}{2} g_{tt,r}
            = \inv{2 (- a^2)} \pdif{1}{r} \\
            &= 0 \: ,
    \end{split}
\end{align}
\end{subequations}
hvor vi har benytte $r \in \{x,\, y,\, z\}$, da $g_{xx} = g_{yy} = g_{zz}$. Christoffelsymbolerne med alle forskellige indeks ($a \ne b \ne c \ne a$) giver $0$, da kun $g_{aa} \ne 0$, f.eks. $\Gamma^r_{r' t} \:\: \forall r' \ne r$. Yderligere vides det, at $\Gamma_{abc} = \Gamma_{acb}$, hvilket giver de resterende Christoffelsymboler.
Dermed har vi at de eneste ikkeforsvindende Christoffelsymboler er
\begin{align}
    \Gamma^x_{tx} &= \Gamma^y_{ty} = \Gamma^z_{tz} = \frac{\dt{a}}{a} \: ,
        \quad \text{og} \quad
    \Gamma^t_{xx} = \Gamma^t_{yy} = \Gamma^t_{zz} = a \dt{a} \: .
\end{align}

Som det næste ønsker vi at beregne Riccitensoren, hvilken er givet ved Riemanntensoren, som igen er givet ved Christoffelsymbolerne, som
\begin{align}
    R_{ab} &= R^d_{adb}
        = \Gamma^d_{ab,d} - \Gamma^d_{ad,b} + \Gamma^d_{ed} \Gamma^e_{ab} - \Gamma^d_{eb} \Gamma^e_{ad} \: ,
\end{align}
hvorved vi får
\begin{subequations}
\begin{align}
    %%%%%%%%%%%%%%%%%%%%%%%%%%%%%%%%%%%%%%%%%%%%%%%%%%
    % tt
    %%%%%%%%%%%%%%%%%%%%%%%%%%%%%%%%%%%%%%%%%%%%%%%%%%
    \begin{split}
        R^t_t &= g^{tt} R_{tt}  \\
            &= g^{tt} R^d_{t d t} \\
            &= g^{tt} \left( \Gamma^d_{tt,d} - \Gamma^d_{td,t} + \Gamma^d_{ed} \Gamma^e_{tt} - \Gamma^d_{et} \Gamma^e_{td} \right) \\
            &= g^{tt} \Bigg( \left[ \Gamma^t_{tt,t} + 3\Gamma^r_{tt,r} \right] - \left[ \Gamma^t_{tt,t} + 3\Gamma^r_{tr,t} \right] \\
                &\qquad\quad + \left[ \left\{ \Gamma^t_{tt} \Gamma^t_{tt} + 3 \Gamma^t_{rt} \Gamma^r_{tt} \right\} + 3 \left\{ \Gamma^r_{tr} \Gamma^t_{tt} + \sum_{r_i} \Gamma^r_{r_i r} \Gamma^{r_i}_{tt} \right\} \right] \\
                &\qquad\quad - \left[ \left\{ \Gamma^t_{tt} \Gamma^t_{tt} + 3 \Gamma^t_{rt} \Gamma^r_{tt} \right\} + 3 \left\{ \Gamma^r_{tt} \Gamma^t_{tr} + \sum_{r_i} \Gamma^r_{r_i t} \Gamma^{r_i}_{tr} \right\}  \right] \Bigg) \\
            &= g^{tt} \left( - 3\Gamma^r_{tr,t} - 3 \Gamma^r_{rt} \Gamma^r_{tr} \right) \\
            &= \inv{1} \left[ - 3 \pfrac{\dt{a}}{a}_{,t} - 3 \pfrac{\dt{a}}{a} \pfrac{\dt{a}}{a} \right] \\
            &= - 3 \left[ \frac{\ddt{a}}{a} - \frac{\dt{a}^2}{a^2} + \frac{\dt{a}^2}{a^2} \right] \\
            &= - 3 \frac{\ddt{a}}{a} \: .
    \end{split} \\
    %%%%%%%%%%%%%%%%%%%%%%%%%%%%%%%%%%%%%%%%%%%%%%%%%%
    % rr
    %%%%%%%%%%%%%%%%%%%%%%%%%%%%%%%%%%%%%%%%%%%%%%%%%%
    \begin{split}
        R^r_r &= g^{rr} R_{rr} \\
            &= g^{rr} R^d_{rdr} \\
            &= g^{rr} \left( \Gamma^d_{rr,d} - \Gamma^d_{rd,r} + \Gamma^d_{ed} \Gamma^e_{rr} - \Gamma^d_{er} \Gamma^e_{rd} \right) \\
            &= g^{rr} \Bigg( \left[ \Gamma^t_{rr,t} + \sum_{r_i} \Gamma^{r_i}_{rr,r_i} \right] - \left[ \Gamma^t_{rt,r} + \sum_{r_i} \Gamma^{r_i}_{r r_i,r} \right] \\
                &\qquad\quad + \left[ \left\{ \Gamma^t_{tt} \Gamma^t_{rr} + \sum_{r_i} \Gamma^t_{r_i t} \Gamma^{r_i}_{rr} \right\} + \sum_{r_i} \left\{ \Gamma^{r_i}_{t r_i} \Gamma^t_{rr} + \sum_{r'} \Gamma^{r_i}_{r' r_i} \Gamma^{r'}_{rr} \right\} \right] \\
                &\qquad\quad - \left[ \left\{ \Gamma^t_{tr} \Gamma^t_{rt} + \sum_{r_i} \Gamma^t_{r_i r} \Gamma^{r_i}_{rt} \right\} + \sum_{r_i} \left\{ \Gamma^{r_i}_{tr} \Gamma^t_{r r_i} + \sum_{r'} \Gamma^{r_i}_{r' r} \Gamma^{r'}_{r r_i} \right\} \right] \Bigg) \\
            &= g^{rr} \left( \Gamma^t_{rr,t} + \sum_{r_i} \left[ \Gamma^{r_i}_{t r_i} \Gamma^t_{rr} \right] - \Gamma^t_{rr} \Gamma^r_{rt} - \Gamma^r_{tr} \Gamma^t_{rr} \right) \\
            &= g^{rr} \left( \Gamma^t_{rr,t} + 3 \Gamma^r_{tr} \Gamma^t_{rr} - \Gamma^t_{rr} \Gamma^r_{rt} - \Gamma^r_{tr} \Gamma^t_{rr} \right) \\
            &= g^{rr} \left( \Gamma^t_{rr,t} + \Gamma^r_{tr} \Gamma^t_{rr} \right) \\
            &= \inv{-a^2} \left[ \left( a \dt{a} \right)_{,t} + \pfrac{\dt{a}}{a} \left( a \dt{a} \right) \right] \\
            &= \inv{-a^2} \left[ \left( \dt{a}^2 + a \ddt{a} \right) + \dt{a}^2 \right] \\
            &= - \frac{\ddt{a}}{a} - 2\frac{\dt{a}^2}{a^2} \: .
    \end{split}
\end{align}
\end{subequations}

Vi kan nu beregne Ricciskalaren ved
\begin{align}
    R &= g^{ab} R_{ab} \: ,
\end{align}
hvorved vi får
\begin{align}
\begin{split}
    R &= g^{aa} R_{aa} \\
        &= g^{tt} R_{tt} + 3 g^{rr} R_{rr} \\
        &= R^t_t + 3 R^r_r \\
        &= 3 \frac{\ddt{a}}{a} + 3 \left[ - \frac{\ddt{a}}{a} - 2 \frac{\dt{a}^2}{a^2} \right] \\
        &= - 3 \frac{\ddt{a}}{a} - 3\frac{\ddt{a}}{a} - 6\frac{\dt{a}^2}{a^2} \\
        &= - 6 \left[ \frac{\ddt{a}}{a} + \frac{\dt{a}^2}{a^2} \right] \: ,
\end{split}
\end{align}
da kun $g^{aa} \ne 0$.

Dermed har vi, at Riccitensoren er
\begin{align}
    R^t_t &= - 3 \frac{\ddt{a}}{a} \: ,
        \quad \text{og} \quad
    R^x_x = R^y_y = R^z_z = - \frac{\ddt{a}}{a} - 2 \frac{\dt{a}^2}{a^2} \: ,
\end{align}
og Ricciskalaren er
\begin{align}
    R &= - 6 \left[ \frac{\ddt{a}}{a} + \frac{\dt{a}^2}{a^2} \right] \: .
\end{align}

Einsteinligningen er givet ved \cite[ligning 33]{ugeseddel7},
\begin{align}
    R_{ab} - \inv{2} R g_{ab} &= \kappa T_{ab} \: ,
\end{align}
hvorved ${}^t_t$-komponenten bliver
\begin{align}
    R^t_t - g^{tt} \inv{2} R g_{tt}
        &= g^{tt} R_{tt} - g^{tt} \inv{2} R g_{tt}
        = g^{tt} \kappa T_{tt}
        = \kappa T^t_t \: ,
\end{align}
og siden vi ønsker denne nedskrevet for et univers som værende en perfekt væske, da skal vi se på, hvordan Einsteinligningen ser ud i Friedmannuniverset. Stress-energitensoren for en perfekt væske er
\begin{align}
    T_{ab} &= (\epsilon + p) u_a u_b - p g_{ab} \: ,
\end{align}
hvor $\epsilon$ er restenergidensiteten, $p$ er trykket og $u_a = \dd x_a / \dd s$. Vi ved at
\begin{align}
\begin{split}
    \dd s^2 &= g_{ab}\, \dd x^a\, \dd x^b
        = g^{ab}\, \dd x_a\, \dd x_b
        = \inv{g_{ab}}\, \dd x_a\, \dd x_b \\
    \Rightarrow
    1 &= \inv{g_{ab}}\, \dif{x_a}{s}\, \dif{x_b}{s}
        = \inv{g_{ab}}\, u_a u_b \\
    \Rightarrow
    g_{ab} &= u_a u_b \: ,
\end{split}
\end{align}
hvorfor
\begin{align}
    T_{ab} &= (\epsilon + p) g_{ab} - p g_{ab}
        = \epsilon g_{ab} \: ,
\end{align}
og indsætter vi dette i Einsteinligningen, så får vi
\begin{align}
\begin{split}
    R^t_t - g^{tt} \inv{2} R g_{tt} &= g^{tt} \kappa \epsilon g_{tt} \\
        \Rightarrow \kappa \epsilon &= - 3 \frac{\ddt{a}}{a} - \inv{2} \left( - 6 \left[ \frac{\ddt{a}}{a} + \frac{\dt{a}^2}{a^2} \right] \right) \\
            &= - 3 \frac{\ddt{a}}{a} + 3 \left[ \frac{\ddt{a}}{a} + \frac{\dt{a}^2}{a^2} \right] \\
            &= - 3 \frac{\ddt{a}}{a} + 3 \frac{\ddt{a}}{a} + 3 \frac{\dt{a}^2}{a^2} \\
            &= 3 \frac{\dt{a}^2}{a^2} \: .
\end{split}
\end{align}
${}^t_t$-komponenten af Einsteinligningen for en perfekt væske er dermed
\begin{align}
    3 \frac{\dt{a}^2}{a^2} &= \kappa \epsilon \: .
\end{align}

Som det næste benytter vi energibevarelsesligningen, som siger at
\begin{align}
    \dd E &= - p\, \dd V \: .
\end{align}
Energien af et volumen er $E = \epsilon V$, hvorved vi får
\begin{align}
\begin{split}
    - p\, \dd V &= \dd E
        = \dd (\epsilon V)
        = V\, \dd \epsilon + \epsilon\, \dd V \\
    \Rightarrow
    \frac{\dd V}{V} &= - \frac{\dd \epsilon}{\epsilon + p} \: ,
\end{split}
\end{align}
og betragter vi nu et volumen af størrelse $V = a^3$, da får vi $\dd V = 3 a^2\, \dd a$, så
\begin{align}
\begin{split}
    - \frac{\dd \epsilon}{\epsilon + p} &= \frac{\dd V}{V}
        = \frac{3 a^2\, \dd a}{a^3}
        = \frac{3}{a}\, \dd a \\
    \Rightarrow
    - \int \frac{\dd \epsilon}{\epsilon + p} &= \int \frac{3}{a}\, \dd a
        = 3 \int \inv{a}\, \dd a
        = 3 \ln(a) \: .
\end{split}
\end{align}
Energibevarelsesligningen er altså
\begin{align}
    3 \ln(a) = - \int \frac{\dd \epsilon}{(\epsilon + p)} \: .
\end{align}
\\

Vi betragter nu et massedomineret univers.


==============================================
==============================================

\begin{enumerate}
    \item Integrer ligningerne for et massedomineret univers ($p = 0$, $\epsilon = \mu$)
        \begin{align}
            \mu a^3 &= \text{konst} \: ,
                \quad \text{og} \quad
            a \propto t^{2/3} \: .
        \end{align}
    \item Integrer ligningerne for et strålingsdomineret univers ($p = \epsilon / 3$)
        \begin{align}
            \epsilon a^4 &= \text{konst} \: ,
                \quad \text{og} \quad
            a \propto t^{1/2} \: .
        \end{align}
\end{enumerate}



%%%%%%%%%%%%%%%%%%%%%%%%%%%%%%%%%%%%%%%%%%%%%%%%%%%%%%%%%%%%%%%%%%%%%%%%%%%%%%%%

\subsection{Opgave 5 -- Volumen af åbent og lukket univers}
\setcounter{subsection}{5}
\setcounter{equation}{0}

Beregn volumenet af et åbent og et lukket univers.

%%%%%%%%%%%%%%%%%%%%%%%%%

\subsubsection*{Besvarelse}

Metrikken af hver hyperflade i det lukkede univers er
\begin{align}
    \text{Lukket:} \qquad
    \dd \sigma^2 &= a^2 \left[ \dd\chi^2 + \sin^2(\chi) \{ \dd \theta^2 + \sin^2(\theta) \dd \phi^2 \} \right] \: ,
\end{align}
hvor $\chi \in [0,\, \pi]$, $\theta \in [0,\, \pi]$ og $\phi \in [0,\, 2\pi]$.

Overfladen af hypersfæren svarer til volumenet af det lukkede univers. Trevolumenet er dermed fundet ved at integrere over alle overflader af tosfærerne fra ''nordpolen'' af hypersfæren til ''sydpolen''
\begin{align}
\begin{split}
    V_{lukket} &= \int [a \dd \chi] [a \sin(\chi) \dd \theta] [a \sin(\chi) \sin(\theta) \dd \phi] \\
        &= a^3 \int_0^\pi \sin^2(\chi) \dd \chi \int_0^\pi \sin(\theta) \dd\theta \int_0^{2\pi} \dd\phi \\
        &= 4 \pi a^3 \int_0^\pi \sin^2(\chi) \dd \chi \\
        &= 4 \pi^2 a^3 \: .
\end{split}
\end{align}
\\

Den samme udregning kan foretager i det åbne univers, hvor metrikken af hver hyperflade er
\begin{align}
    \text{Åbent:} \qquad
    \dd \sigma^2 &= a^2 \left[ \dd\chi^2 + \sinh^2(\chi) \{ \dd \theta^2 + \sin^2(\theta) \dd \phi^2 \} \right] \: ,
\end{align}
hvor $\chi \in [0,\, \infty[$, $\theta \in [0,\, \pi]$ og $\phi \in [0,\, 2\pi]$. Denne ændring for definitionsmængden af $\chi$ betyder, at integralet vil divergere, siden $\sinh(\chi) \rightarrow \infty$ når $\chi \rightarrow \infty$. Trevolumenet af det åbne univers er dermed uendelig.



%%%%%%%%%%%%%%%%%%%%%%%%%%%%%%%%%%%%%%%%%%%%%%%%%%%%%%%%%%%%%%%%%%%%%%%%%%%%%%%%%%%%%

\end{document}