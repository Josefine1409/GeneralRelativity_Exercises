\documentclass[../main.tex]{subfiles}

\begin{document}

%%%%%%%%%%%%%%%%%%%%%%%%%%%%%%%%%%%%%%%%%%%%%%%%%%%%%%%%%%%%%%%%%%%%%%%%%%%%%%%%%%%%%

\section{Uge 12 -- Friedmanunivers (FLRW-univers)}
\setcounter{section}{12}


%%%%%%%%%%%%%%%%%%%%%%%%%%%%%%%%%%%%%%%%%%%%%%%%%%%%%%%%%%%%%%%%%%%%%%%%%%%%%%%%

\subsection{Opgave 1 -- Omskriv ${}^\eta_\eta$-Friedmanligningen til ${}^t_t$-versionen}
\setcounter{subsection}{1}
\setcounter{equation}{0}

Vores ${}^\eta_\eta$-Friedmanligning er skrevet for den $\eta$-afledede, $a' \doteq \dd a/\dd\eta$. Omskriv den (som i Wikeperdiaartiklen\footnote{Muligvis denne artikel: \url{https://en.wikipedia.org/wiki/Friedmann_equations}.}) for den $t$-afledede, $\dt{a} = \dd a/\dd t$, i stedet.

%%%%%%%%%%%%%%%%%%%%%%%%%

\subsubsection*{Besvarelse}

${}^\eta_\eta$-komponenten af Friedmanligningen for det lukkede univers er \cite[ligning 15]{ugeseddel12}
\begin{align}
    \frac{3}{a^4} \left( a^2 + a'^2 \right) &= \kappa \epsilon \: .
\end{align}
Vi ved også, at $\dd t = a \dd \eta$, \cite[ligning 11]{ugeseddel12}, hvorfor
\begin{align}
    a' &= \dif{a}{\eta}
        = \dif{a}{t}\, \dif{t}{\eta}
        = \dot{a} a \: ,
\end{align}
hvorfor vi får ${}^t_t$-komponenten af Friedmanligningen for det lukkede univers til
\begin{align}
\begin{split}
    \kappa \epsilon &= \frac{3}{a^4} \left( a^2 + a'^2 \right) \\
        &= \frac{3}{a^4} \left( a^2 + \left[ \dt{a} a \right]^2 \right) \\
        &= \frac{3}{a^4} \left( a^2 + a^2 \dt{a}^2 \right) \\
        &= \frac{3 a^2}{a^4} \left( 1 + \dt{a}^2 \right) \\
        &= \frac{3}{a^2} \left( 1 + \dt{a}^2 \right) \: . \\
    \Rightarrow
    \pfrac{\dt{a}}{a}^2 &= \frac{\kappa \epsilon}{3} - \inv{a^2} \: .
\end{split}
\end{align}



%%%%%%%%%%%%%%%%%%%%%%%%%%%%%%%%%%%%%%%%%%%%%%%%%%%%%%%%%%%%%%%%%%%%%%%%%%%%%%%%

\subsection{Opgave 2 -- Riccitensor i Friedmankoordinater for lukket univers}
\setcounter{subsection}{2}
\setcounter{equation}{0}

Beregn manuelt Riccitensoren i Friedmankoordinater for et lukket univers.

%%%%%%%%%%%%%%%%%%%%%%%%%

\subsubsection*{Besvarelse}

\ldots



%%%%%%%%%%%%%%%%%%%%%%%%%%%%%%%%%%%%%%%%%%%%%%%%%%%%%%%%%%%%%%%%%%%%%%%%%%%%%%%%

\subsection{Opgave 3 -- Energibevarelse fra feltligninger}
\setcounter{subsection}{3}
\setcounter{equation}{0}

Argumentér for at energibevarelsensligningen,
\begin{align}
    (\epsilon a^3)' + p(a^3)' &= 0 \: ,
\end{align}
følger fra feltligningerne
\begin{align}
    \frac{3}{a^4} \left( a^2 + a'^2 \right) &= \kappa \epsilon \: , \quad \text{og} \label{eq:Uge12_Opg3_FieldEquation1} \\
    \inv{a^4} \left( a^2 + 2aa'' - a'^2 \right) &= - \kappa p \: . \label{eq:Uge12_Opg3_FieldEquation2}
\end{align}

%%%%%%%%%%%%%%%%%%%%%%%%%

\subsubsection*{Besvarelse}

Vi betragter først den første feltligning, \cref{eq:Uge12_Opg3_FieldEquation1}, hvori vi isolerer $\epsilon a^3$
\begin{align}
    \kappa \epsilon &= \frac{3}{a^4} \left( a^2 + a'^2 \right)
        \quad \Rightarrow \quad
    \epsilon a^3 = \frac{3}{a \kappa} \left( a^2 + a'^2 \right) \: ,
\end{align}
hvorefter vi nu differentierer dette med hensyn til $\eta$
\begin{align} \label{eq:Uge12_Opg3_FirstPart}
\begin{split}
    (\epsilon a^3)' &= \dif{}{\eta} \left[ \frac{3}{a \kappa} \left( a^2 + a'^2 \right) \right] \\
        &= \frac{3}{\kappa} \dif{}{\eta} \left( a + \frac{a'^2}{a} \right) \\
        &= \frac{3}{\kappa} \left( a' + \frac{2a'a''}{a} - \frac{a'^2}{a^2} a' \right) \\
        &= \frac{3 a'}{\kappa} + \frac{6a'a''}{a \kappa} - \frac{3 a'^3}{a^2 \kappa} \: .
\end{split}
\end{align}

Betragter vi nu den anden feltligning, \cref{eq:Uge12_Opg3_FieldEquation2}, hvori vi isolerer $-p$, så får vi
\begin{align}
    - \kappa p &= \inv{a^4} \left( a^2 + 2aa'' - a'^2 \right)
        \quad \Rightarrow \quad
    - p = \inv{a^4 \kappa} \left( a^2 + 2aa'' - a'^2 \right) \: ,
\end{align}
hvorefter vi nu ganger dette på $(a^3)'$, altså ganget på $a^3$ differentieret med hensyn til $\eta$,
\begin{align} \label{eq:Uge12_Opg3_SecondPart}
\begin{split}
    - p (a^3)' &= \inv{a^4 \kappa} \left( a^2 + 2aa'' - a'^2 \right) \dif{a^3}{\eta} \\
        &= \inv{a^4 \kappa} \left( a^2 + 2aa'' - a'^2 \right) 3a^2 a' \\
        &= \frac{3 a'}{\kappa} + \frac{6a'a''}{a \kappa} - \frac{3 a'^3}{a^2 \kappa} \: .
\end{split}
\end{align}

Det kan nu ses, at vi kan sætte \cref{eq:Uge12_Opg3_FirstPart} lig \cref{eq:Uge12_Opg3_SecondPart}, hvorved vi får
\begin{align}
\begin{split}
    (\epsilon a^3)' &= \frac{3 a'}{\kappa} + \frac{6a'a''}{a \kappa} - \frac{3 a'^3}{a^2 \kappa} = - p (a^3)' \\
    \Rightarrow 0 &= (\epsilon a^3)' + p (a^3)' \: ,
\end{split}
\end{align}
hvilket er energibevarelsesligningen, som vi skulle vise.



%%%%%%%%%%%%%%%%%%%%%%%%%%%%%%%%%%%%%%%%%%%%%%%%%%%%%%%%%%%%%%%%%%%%%%%%%%%%%%%%

\subsection{Opgave 4 -- Tidsudvikling af masse- og strålingsdominerede Euklidiske universer}
\setcounter{subsection}{4}
\setcounter{equation}{0}

Betragt et fladt (Euklidisk) og isotopt univers med metrikken
\begin{align}
    \dd s^2 &= \dd t^2 - a^2(t) \left( \dd x^2 + \dd y^2 + \dd z^2 \right) \: ,
\end{align}
og undersøg dets tidslige udvikling for masse- og strålingsdominerede universer.

Hints:
\begin{enumerate}
    \item Beregn Christoffelsymbolerne
        \begin{align}
            \Gamma^x_{tx} &= \Gamma^y_{ty} = \Gamma^z_{tz} = \frac{\dt{a}}{a} \: ,
                \quad \text{og} \quad
            \Gamma^t_{xx} = \Gamma^t_{yy} = \Gamma^t_{zz} = a \dt{a} \: .
        \end{align}
    \item Beregn Riccitensoren og Ricciskalaren
        \begin{align}
            R^t_t &= - 3 \frac{\ddt{a}}{a} \: ,
                \quad \text{og} \quad
            R^x_x = R^y_y = R^z_z = - \frac{\ddt{a}}{a} - 2 \frac{\dt{a}^2}{a^2} \: .
        \end{align}
    \item Nedskriv ${}^t_t$-komponenten af Einsteinligningen med en perfekt væske
        \begin{align}
            3 \frac{\dt{a}^2}{a^2} &= \kappa \epsilon \: .
        \end{align}
    \item Nedskriv energibevarelsesligningen
        \begin{align}
            \frac{\dd V}{V} &= - \frac{\dd \epsilon}{(\epsilon + p)}
                \quad \Rightarrow \quad
            3 \ln(a) = - \int \frac{\dd \epsilon}{(\epsilon + p)} \: .
        \end{align}
    \item Medtag (eng: integrate) ligningerne for et massedomineret univers ($p = 0$, $\epsilon = \mu$)
        \begin{align}
            \mu a^3 &= \text{konst} \: ,
                \quad \text{og} \quad
            a \propto t^{2/3} \: .
        \end{align}
    \item Medtag (eng: integrate) ligningerne for et strålingsdomineret univers ($p = \epsilon / 3$)
        \begin{align}
            \epsilon a^4 &= \text{konst} \: ,
                \quad \text{og} \quad
            a \propto t^{1/2} \: .
        \end{align}
\end{enumerate}

%%%%%%%%%%%%%%%%%%%%%%%%%

\subsubsection*{Besvarelse}

\ldots



%%%%%%%%%%%%%%%%%%%%%%%%%%%%%%%%%%%%%%%%%%%%%%%%%%%%%%%%%%%%%%%%%%%%%%%%%%%%%%%%

\subsection{Opgave 5 -- Volumen af åbent og lukket univers}
\setcounter{subsection}{5}
\setcounter{equation}{0}

Beregn volumenet af et åbent og et lukket univers.

%%%%%%%%%%%%%%%%%%%%%%%%%

\subsubsection*{Besvarelse}

Metrikken af hver hyperflade i det lukkede univers er
\begin{align}
    \text{Lukket:} \qquad
    \dd \sigma^2 &= a^2 \left[ \dd\chi^2 + \sin^2(\chi) \{ \dd \theta^2 + \sin^2(\theta) \dd \phi^2 \} \right] \: ,
\end{align}
hvor $\chi \in [0,\, \pi]$, $\theta \in [0,\, \pi]$ og $\phi \in [0,\, 2\pi]$.

Overfladen af hypersfæren svarer til volumenet af det lukkede univers. Trevolumenet er dermed fundet ved at integrere over alle overflader af tosfærerne fra ''nordpolen'' af hypersfæren til ''sydpolen''
\begin{align}
\begin{split}
    V_{lukket} &= \int [a \dd \chi] [a \sin(\chi) \dd \theta] [a \sin(\chi) \sin(\theta) \dd \phi] \\
        &= a^3 \int_0^\pi \sin^2(\chi) \dd \chi \int_0^\pi \sin(\theta) \dd\theta \int_0^{2\pi} \dd\phi \\
        &= 4 \pi a^3 \int_0^\pi \sin^2(\chi) \dd \chi \\
        &= 4 \pi^2 a^3 \: .
\end{split}
\end{align}
\\

Den samme udregning kan foretager i det åbne univers, hvor metrikken af hver hyperflade er
\begin{align}
    \text{Åbent:} \qquad
    \dd \sigma^2 &= a^2 \left[ \dd\chi^2 + \sinh^2(\chi) \{ \dd \theta^2 + \sin^2(\theta) \dd \phi^2 \} \right] \: ,
\end{align}
hvor $\chi \in [0,\, \infty[$, $\theta \in [0,\, \pi]$ og $\phi \in [0,\, 2\pi]$. Denne ændring for definitionsmængden af $\chi$ betyder, at integralet vil divergere, siden $\sinh(\chi) \rightarrow \infty$ når $\chi \rightarrow \infty$. Trevolumenet af det åbne univers er dermed uendelig.



%%%%%%%%%%%%%%%%%%%%%%%%%%%%%%%%%%%%%%%%%%%%%%%%%%%%%%%%%%%%%%%%%%%%%%%%%%%%%%%%%%%%%

\end{document}