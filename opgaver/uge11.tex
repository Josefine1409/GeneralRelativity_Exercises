\documentclass[../main.tex]{subfiles}

\begin{document}

%%%%%%%%%%%%%%%%%%%%%%%%%%%%%%%%%%%%%%%%%%%%%%%%%%%%%%%%%%%%%%%%%%%%%%%%%%%%%%%%%%%%%

\section{Uge 11 -- Bevægelse i Schwarzschildmetrikken}
\setcounter{section}{11}


%%%%%%%%%%%%%%%%%%%%%%%%%%%%%%%%%%%%%%%%%%%%%%%%%%%%%%%%%%%%%%%%%%%%%%%%%%%%%%%%

\subsection{(M) Opgave 1 -- Bevægelsesligning for radiært frit fald mod sort hul fra uendelig}
\setcounter{subsection}{1}
\setcounter{equation}{0}

Udled bevægelsesligningen
\begin{align} \label{eq:Uge11_Opg1_EquationOfMotion}
    \dd t &= - \sqf{r}{r_s} \inv{1 - \frac{r_s}{r}}\, \dd r
\end{align}
for et objekt i et radiært frit fald fra uendelig ind mod et sort hul ved at benytte de geodætiske ligninger i Schwarzschildmetrikken.

Hints: Find $\dd t / \dd s$ fra $t$-geodætligningen. Find $\dd r / \dd s$ fra $r$-geodætligningen ved at eliminere $t(s)$ ved brug af udtrykket for metrikken og integrer én gang. Divider de to.

%%%%%%%%%%%%%%%%%%%%%%%%%

\subsubsection*{Besvarelse}

\ldots



%%%%%%%%%%%%%%%%%%%%%%%%%%%%%%%%%%%%%%%%%%%%%%%%%%%%%%%%%%%%%%%%%%%%%%%%%%%%%%%%

\subsection{Opgave 2 -- Fritfaldende objekt i Schwarzschildmetrikken}
\setcounter{subsection}{2}
\setcounter{equation}{0}

Et objekt falder frit og radiært mod centrum i Schwarzschildmetrikken. Hvad er dets koordinathastighed $\dd r / \dd t$ ved radius $r$? Hvad er den lokalmålte hastighed for samme $r$?

Hint: I Schwarzschildmetrikken er den lokalmålte radiære afstand givet som $\dd \hat{r}^2 = (1 - 2M/r)^{-1} \dd r^2$ og den lokalmålte tid er givet som $\dd \hat{t}^2 = (1 - 2M/r) \dd t^2$

%%%%%%%%%%%%%%%%%%%%%%%%%

\subsubsection*{Besvarelse}

Beægelsesligningen for et objekt i frit og radiært fald mod centrum i Schwarzschildmetrikken er \cref{eq:Uge11_Opg1_EquationOfMotion},
\begin{align} 
    \dd t &= - \sqf{r}{r_s} \inv{1 - \frac{r_s}{r}}\, \dd r \: ,
\end{align}
og isolerer vi nu koordinathastigheden $\dd r / \dd t$ fås
\begin{align}
    \dif{r}{t} &= -\, \inv{\sqrt{\dfrac{r}{r_s}}\, \inv{1 - \frac{r_s}{r}}}
        = -\, \dfrac{1 - \frac{r_s}{r}}{\sqrt{\dfrac{r}{r_s}}}
        = - \left( 1 - \frac{r_s}{r} \right) \sqf{r_s}{r}
        = \left( \frac{r_s}{r} - 1 \right) \sqf{r_s}{r} \: .
\end{align}
\\

For nu at finde den lokaltmålte hastighed ved samme $r$, da benytter vi at den lokaltmålte radiære afstand og lokaltmålte tid er givet som
\begin{align}
    \dd \hat{r}^2 &= \left( 1 - \frac{2M}{r} \right)^{-1} \dd r^2
        \qquad \text{og} \qquad
    \dd \hat{t}^2 = \left( 1 - \frac{2M}{r} \right) \dd t^2 \: ,
\end{align}
hvori vi isolerer $\dd r$ og $\dd t$,
\begin{align}
    \begin{split}
        \dd \hat{t}^2 &= \left( 1 - \frac{2M}{r} \right) \dd t^2
        \quad \Rightarrow \quad
        \dd t^2 = \left( 1 - \frac{2M}{r} \right)^{-1} \dd \hat{t}^2
        \quad \Rightarrow \quad
        \dd t = \dfrac{\dd \hat{t}}{\sqrt{1 - \frac{2M}{r}}} \: , \quad \text{og}
    \end{split} \\
    \begin{split}
        \dd \hat{r}^2 &= \left( 1 - \frac{2M}{r} \right)^{-1} \dd r^2
        \quad \Rightarrow \quad
        \dd r^2 = \left( 1 - \frac{2M}{r} \right) \dd \hat{r}^2
        \quad \Rightarrow \quad
        \dd r = \sqrt{1 - \frac{2M}{r}}\, \dd \hat{r} \: ,
    \end{split}
\end{align}
hvorefter disse indsættes i formlen for koordinathastigheden, og der isoleres for $\dd \hat{r} / \dd \hat{t}$
\begin{align}
\begin{split}
    \left( \frac{r_s}{r} - 1 \right) \sqf{r_s}{r} &= \dif{r}{t} \\
        &= \dfrac{\sqrt{1 - \frac{2M}{r}}\, \dd \hat{r}}{\inv{\sqrt{1 - \frac{2M}{r}}}\,\dd \hat{t}} \\
        &= \left( 1 - \frac{2M}{r} \right) \dif{\hat{r}}{\hat{t}} \\
    \Rightarrow
    \dif{\hat{r}}{\hat{t}} &= \left( \frac{r_s}{r} - 1 \right) \sqf{r_s}{r} \left( 1 - \frac{2M}{r} \right)^{-1} \\
        &= \left( \frac{r_s}{r} - 1 \right) \sqf{r_s}{r} \left( 1 - \frac{r_s}{r} \right)^{-1} \\
        &= \left( \frac{r_s}{r} - 1 \right) \sqf{r_s}{r} \left( - \left[ \frac{r_s}{r} -1 \right] \right)^{-1} \\
        &= - \sqf{r_s}{r} \: ,
\end{split}
\end{align}
da vi har sat $G = c = 1$, hvorfor Schwarzschildradien er $r_s = 2GM/c^2 = 2M$.
Dermed er den lokaltmålte koordinathastighed
\begin{align}
    \dif{\hat{r}}{\hat{t}} &= - \sqf{r_s}{r}
\end{align}
for afstanden $r$.



%%%%%%%%%%%%%%%%%%%%%%%%%%%%%%%%%%%%%%%%%%%%%%%%%%%%%%%%%%%%%%%%%%%%%%%%%%%%%%%%

\subsection{Opgave 3 -- Masse af sort hul grundet rødforskydning}
\setcounter{subsection}{3}
\setcounter{equation}{0}

En radiosender falder frit radiært ind mod et sort hul. Som radiosenderen nærmer sig den gravitationelle radius måles dens radiosignal af en udenforstående observatør til at være rødforskudt med $\omega = \omega_0 \pexp{-\lambda t}$. Estimér massen af det sorte hul ud fra det målte $\lambda$.

Hints: $\omega_0 = \omega / \sqrt{g_{00}}$, \cite[ligning 13]{ugeseddel10}, og $r - r_s = (r_1 - r_s)\pexp{-[t - t_1]/r_s}$, \cite[ligning 7]{ugeseddel11}.

%%%%%%%%%%%%%%%%%%%%%%%%%

\subsubsection*{Besvarelse}

Vi betragter de to ligninger
\begin{align}
    \omega_0 &= \frac{\omega}{\sqrt{g_{00}}}
        \quad \text{og} \quad
    r - r_s = (r_0 - r_s)\pexp{-[t - t_0]/r_s} \: .
\end{align}
hvor $\omega$ er frekvensen målt af en ydre observatør, der bruger verdens koordinater $(t,\, r)$, mens $\omega_0$ er egenfrekvensen, målt af en iagttager, som sidder på selve radioen, og som bruger radioens koordinater $(t_0,\, r_0)$. Sidste ligning gør sig gældende, når $r \gtrsim r_s$, altså at vi er lige ved eller lidt over den gravitationelle radius for det sorte hul.

Siden vi ved, at $g_{00} = 1 - r_s/r$ skriver vi først den første ligning ud
\begin{align}
    \omega &= \omega_0 \sqrt{g_{00}}
        = \omega_0 \sqrt{1 - \frac{r_s}{r}} \: ,
\end{align}
hvilket giver os proportionalitetsfaktoren, som vi vil undersøge videre, da vi vil finde et udtryk for $\lambda$ ved $r_s$ og vi ved at $\omega = \omega_0 \pexp{-\lambda t}$. Kigger vi på denne faktor ser vi, at
\begin{align}
\begin{split}
    \Lim{r}{r_s} \sqrt{1 - \frac{r_s}{r}} &= \Lim{r}{r_s} \sqf{r - r_s}{r} \\
        &= \sqf{(r_0 - r_s) \pexp{-\frac{t - t_0}{r_s}}}{r_s} \\
        &= \sqf{r_0 - r_s}{r_s} \sqrt{\pexp{-\frac{t - t_0}{r_s}}} \\
        &= \sqf{r_0 - r_s}{r_s} \pexp{-\frac{t - t_0}{2 r_s}}
\end{split}
\end{align}
hvor vi har benyttet anden ligning i hintet. Sammenligner vi med $\omega = \omega_0 \pexp{-\lambda t}$ får vi, at
\begin{align}
    \lambda &= \inv{2 r_s} \: ,
\end{align}
og da vi ved, at $r_s = 2GM/c^2$, så har vi, at massen af det sorte hul er relateret til rødforskydningen ved
\begin{align}
    M &= \frac{c^2 r_s}{2 G}
        = \frac{c^2}{4 G \lambda}
        \approx \inv{4\lambda} \: .
\end{align}



%%%%%%%%%%%%%%%%%%%%%%%%%%%%%%%%%%%%%%%%%%%%%%%%%%%%%%%%%%%%%%%%%%%%%%%%%%%%%%%%

\subsection{Opgave 4 -- Egentid for Lemaitreur i sort hul}
\setcounter{subsection}{4}
\setcounter{equation}{0}

Beregn egentiden (eng: proper time) det tager for et Lemaitreur at falde fra den gravitationelle radius til centrum af et sort hul. Specificer svaret i sekunder for et sort hul med masse som Solen.

%%%%%%%%%%%%%%%%%%%%%%%%%

\subsubsection*{Besvarelse}

Egentiden, som det tager for et Lamaitreur at falde fra en radius $r_1$ til en mindre radius $r_2$, er givet ved \cite[ligning 8]{ugeseddel11} (hvor vi har indsat $c$'et, som på ugesedlen ikke er der, da det står i naturlige enheder)
\begin{align}
    \Delta \tau &= \frac{2}{3c} \pfrac{r_1^{3/2} - r_2^{3/2}}{r_s^{1/2}} \: ,
\end{align}
så hvis vi indsætter $r_1$ værende den gravitationelle radius, $r_1 = r_s = 2GM/c^2$, hvor $M$ er massen af det sorte hul, og $r_2$ værende centrum af det sorte hul, $r_2 = 0$, så får vi
\begin{align}
    \Delta \tau &= \frac{2}{3c} \pfrac{r_s^{3/2} - 0^{3/2}}{r_s^{1/2}} 
        = \frac{2}{3c} \pfrac{r_s^{3/2}}{r_s^{1/2}}
        = \frac{2}{3c} r_s
        = \frac{2}{3c}\, \frac{2GM}{c^2}
        = \frac{4GM}{3c^3} \: .
\end{align}

For et sort hul med masse som solen, da er $M = M_{\astrosun} = \SI{1.989E30}{\kilo\gram}$, gravitationskonstanten er $G = \SI{6.67408E-11}{\cubic\meter\per\kilo\gram\per\square\second}$ og lysets hastighed er $c = \SI{2.99792458E8}{\meter\per\second}$ (tallene er fundet på Google/Wikipedia d. 15. januar 2021). Dermed får vi
\begin{align}
    \Delta \tau &= \frac{4GM_{\astrosun}}{3c^3}
        = \frac{4 \cdot \SI{6.67408E-11}{\cubic\meter\per\kilo\gram\per\square\second} \cdot \SI{1.989E30}{\kilo\gram}}{3 \cdot \left(\SI{2.99792458E8}{\meter\per\second}\right)^3}
        = \SI{6.569}{\micro\second} \: .
\end{align}



%%%%%%%%%%%%%%%%%%%%%%%%%%%%%%%%%%%%%%%%%%%%%%%%%%%%%%%%%%%%%%%%%%%%%%%%%%%%%%%%

\subsection{(M) Opgave 5 -- Cirkulær banes stabilitet overfor radiær perturbation}
\setcounter{subsection}{5}
\setcounter{equation}{0}

Vis at en cirkulær planetbane omkring en stjerne i Newtonsk mekanik er stabil overfor små radiære perturbationer. Vis at en cirkulær planetbane i generel relativitetsteori kun er stabil overfor samme perturbation, hvis $r > 6M$.

Hints: Betragt en cirkulær planetbane med en lille perturbation $u = u_0 + \delta u$. Udled ligningen for $\delta u$ til laveste orden. Undersøg om perturbationen forbliver lille eller om den bliver større.

%%%%%%%%%%%%%%%%%%%%%%%%%

\subsubsection*{Besvarelse}

\ldots



%%%%%%%%%%%%%%%%%%%%%%%%%%%%%%%%%%%%%%%%%%%%%%%%%%%%%%%%%%%%%%%%%%%%%%%%%%%%%%%%

\subsection{(S) Opgave 6 -- Stabilitet af ækvatorial bane i Schwarzschildmetrikken}
\setcounter{subsection}{6}
\setcounter{equation}{0}

Vi at ækvatoriale baner i Schwarzschildmetrikken er stabile overfor små perturbationer i radius. Betragt for simplicitet en cirkulær bane.

Hint: Betragt en ækvatorial bane med lille perturbation, $\theta = \pi/2 + \delta \theta$. Udled ligningen for $\delta \theta$ til laveste orden. Vis at perturbationen forbliver lille. Fortolk løsningen.

%%%%%%%%%%%%%%%%%%%%%%%%%

\subsubsection*{Besvarelse}

Vi betragter en lille perturbation til en ækvatorial bane
\begin{align}
    \theta &= \frac{\pi}{2} + \delta\theta \: ,
\end{align}
hvor $\delta \theta \ll 1$. Generelt er det bevarede impulsmoment
\begin{align}
    J^2 &= r^2 \left( g^{\theta\theta} u_\theta^2 + g^{\phi\phi} u_\phi^2 \right)
        = \left[ u_\theta^2 + \inv{\sin^2(\theta)} u_\phi^2 \right] \: ,
\end{align}
hvor det er blevet benyttet, at $g^{\theta\theta} = 1/r^2$. Bemærk at i det ævatoriale plan reduceres den bevarede strøm til
\begin{align}
    J^2 &= r^2 u_\phi^2 \: .
\end{align}

\ldots Spørg Rasmus, da det ser ud til, at i (104) kommer nogle r ind, som ikke er der i (102). Hvilken af dem er rigtig? Også er r'et der i (103) men ikke i (102)? \ldots



%%%%%%%%%%%%%%%%%%%%%%%%%%%%%%%%%%%%%%%%%%%%%%%%%%%%%%%%%%%%%%%%%%%%%%%%%%%%%%%%%%%%%

\end{document}