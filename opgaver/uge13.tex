\documentclass[../main.tex]{subfiles}

\begin{document}

%%%%%%%%%%%%%%%%%%%%%%%%%%%%%%%%%%%%%%%%%%%%%%%%%%%%%%%%%%%%%%%%%%%%%%%%%%%%%%%%%%%%%

\section{Uge 13 -- Big Bang}
\setcounter{section}{13}


%%%%%%%%%%%%%%%%%%%%%%%%%%%%%%%%%%%%%%%%%%%%%%%%%%%%%%%%%%%%%%%%%%%%%%%%%%%%%%%%

\subsection{Opgave 1 -- Hastighed af galakse ud fra rødforskydning}
\setcounter{subsection}{1}
\setcounter{equation}{0}

Fortolk den kosmologiske rødforskydning $(\omega_0 - \omega)/\omega_0 = H l$, hvor $l$ er distancen til den rødforskudte galakse, som en Dopplereffekt og beregn hastigheden, som galaksen ser ud til at bevæge sig med ift. observatøren.

%%%%%%%%%%%%%%%%%%%%%%%%%

\subsubsection*{Besvarelse}

Doplereffekten er rødforskydningen grundet den relative hastighed mellem kilden til signalerne og observatøren
\begin{align}
    \frac{\omega - \omega_0}{\omega_0} &= \frac{v_{kilde} - v_{obs}}{v_{med}} \: ,
\end{align}
I vores tilfælde er $v_{med} = 1$ i naturlige enheder, og vi lader $v_{obs} = 0$ for simplicitet. Dermed får vi
\begin{align}
    v_{kilde} &= H l \: ,
\end{align}
hvor $d$ er afstanden til den rødforskudte galakse. Dette er præcis Hubbles lov.



%%%%%%%%%%%%%%%%%%%%%%%%%%%%%%%%%%%%%%%%%%%%%%%%%%%%%%%%%%%%%%%%%%%%%%%%%%%%%%%%

\subsection{Opgave 2 -- $\epsilon a^4 = \text{konst} \: \Rightarrow \: p = \epsilon/3$}
\setcounter{subsection}{2}
\setcounter{equation}{0}

Vis at $\epsilon a^4 = \text{konst}$ svarer til tilstandsligningen $p = \epsilon/3$.

%%%%%%%%%%%%%%%%%%%%%%%%%

\subsubsection*{Besvarelse}

Vi betragter først ligningen $\epsilon a^4 = \mathrm{konst}$, hvor vi isolerer $\epsilon$,
\begin{align}
    \epsilon &= \frac{\mathrm{konst}}{a^4} \: ,
\end{align}
hvorefter vi differentierer denne ligning med hensyn til $a$,
\begin{align}
    \dif{\epsilon}{a} &= \dif{}{a} \pfrac{\mathrm{konst}}{a^4}
        = - \frac{4 \cdot \mathrm{konst}}{a^5}
        =  - \frac{4}{a} \epsilon \: .
\end{align}

Dernæst benytter vi os af energibevarelsesloven
\begin{align}
    - p\, \dd V &= \dd E = \dd (\epsilon V) = V\, \dd \epsilon + \epsilon\, \dd V \: .
\end{align}
Heri kan vi nu indsætte $\dd \epsilon$ som udregnet ovenfor, samt volumenet på $V = a^3$ og $\dd V = 3 a^2\, \dd a$, hvorved vi får
\begin{align}
\begin{split}
    - 3p a^2\, \dd a &= - p\, \dd V
        = V\, \dd \epsilon + \epsilon\, \dd V
        = a^3\, \dd \epsilon + 3 \epsilon a^2\, \dd a \\
    \Rightarrow
    - 3p &= a\, \dif{\epsilon}{a} + 3 \epsilon\, \dif{a}{a}
        = a\, \dif{\epsilon}{a} + 3 \epsilon
        = - a \frac{4}{a} \epsilon + 3 \epsilon
        = - 4 \epsilon + 3 \epsilon
        = - \epsilon \: .
\end{split}
\end{align}
Dermed får vi, at
\begin{align}
    - 3p &= - \epsilon
        \quad \Rightarrow \quad
    p = \frac{\epsilon}{3} \: ,
\end{align}
hvilket vi skule vise.



%%%%%%%%%%%%%%%%%%%%%%%%%%%%%%%%%%%%%%%%%%%%%%%%%%%%%%%%%%%%%%%%%%%%%%%%%%%%%%%%%%%%%

\end{document}