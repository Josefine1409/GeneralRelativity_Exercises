\usepackage{amsmath,amssymb,bm,mathtools}
\usepackage{mathrsfs} % Fancy skrevet bogstaver
% \usepackage{gensymb} % Laver visse symboler (eks. gradetegn) i math- og text-mode.
\usepackage{nicefrac} % Nice fractions

% Differentiering
\newcommand{\dif}[3][]{\frac{\dd^{#1}{#2}}{{\dd{#3}}^{#1}}} % Hård differentiering
    % Differentieres f mht. x n gange skrive \dif[n]{f}{x}. For n=1 lades parentesen være tom, [].
\newcommand{\pdif}[3][]{\frac{\partial^{#1}{#2}}{\partial {#3}^{#1}}} % Partiel differentiering
    % Differentieres f mht. x n gange skrive \(p)dif[n]{f}{x}, hvor p giver bløde afledede. For n=1 lades parentesen være tom, [].
\newcommand{\dt}[1]{\dot{#1}} % Afledt mht. t (dot-notation)
\newcommand{\ddt}[1]{\ddot{#1}} % Dobbelt afledt mht. t (dobbel-dot)
\newcommand{\dr}{\d r} % dr til f.eks. integraler som \dr
\newcommand{\dx}{\d x} % dx til f.eks. integraler som \dx

% Lighedstegn med ovenstående betingelser som \xleq{<betingelse>}
\usepackage{extarrows} % Giver mulighed for at skrive over et lighedstegn
\newcommand{\xleq}{\xlongequal}

% Grænseværdier
\newcommand{\Lim}[2]{\raisebox{0.5ex}{\scalebox{0.8}{$\displaystyle \lim_{{#1} \rightarrow {#2}}\;$}}} % Limit med limits under "lim"
\newcommand{\Liminf}[1]{\raisebox{0.5ex}{\scalebox{0.8}{$\displaystyle \lim_{{#1} \rightarrow \infty}\;$}}} % Limit gående mod uendelig
\newcommand{\Limnul}[1]{\raisebox{0.5ex}{\scalebox{0.8}{$\displaystyle \lim_{{#1} \rightarrow 0}\;$}}} % Limit gående mod 0

% Til Euler-Lagrange ligningen
\newcommand{\el}[1]{\dif{t}{}\left(\pdif{\dt{#1}}{L}\right)}

% Uendelige integraler
\newcommand{\intInf}{\int_{-\infty}^{\infty}} % int_{-inf}^{inf}
\newcommand{\intNul}{\int_0^{\infty}} % int_{0}^{inf}

% Vektorer
\newcommand{\TwoRowMat}[2]{\begin{bmatrix} #1 \\ #2 \end{bmatrix}} % Matrix med 2 rækker
\newcommand{\ThreeRowMat}[3]{\begin{bmatrix} #1 \\ #2 \\ #3 \end{bmatrix}} % Matrix med 3 rækker
\newcommand{\FourRowMat}[4]{\begin{bmatrix} #1 \\ #2 \\ #3 \\ #4 \end{bmatrix}} % Matrix med 4 rækker
\let\vaccent=\v % Omdøb \v{} til \vaccent{} så \v{} er en fri kommando
\renewcommand{\v}[1]{\mathbf{#1}} % Vektor med fed
\newcommand{\vv}[1]{\vec{\mathbf{#1}}} % Vektor med fed og pil
\newcommand{\gv}[1]{{\boldsymbol{#1}}} % Vektor med græske bogstaver (fed)
\newcommand{\vgv}[1]{\Vec{\gv{#1}}} % Vektor med græske bogstaver (fed og pil)
\newcommand{\hatvec}[1]{\hat{\mathbf{#1}}} % Hatvektor
\newcommand{\ihat}{\mathbf{\hat{\imath}}} % Enhedsvektor i
\newcommand{\jhat}{\mathbf{\hat{\jmath}}} % Enhedsvektor j
\newcommand{\khat}{\mathbf{\hat{k}}} % Enhedsvektor k
\newcommand{\rhat}{\mathbf{\hat{r}}} % Enhedsvektor r
\newcommand{\phihat}{\mathbf{\hat{\phi}}} % Enhedsvektor phi
\newcommand{\thetahat}{\mathbf{\hat{\theta}}} % Enhedsvektor theta
\newcommand{\xhat}{\mathbf{\hat{x}}} % Enhedsvektor x
\newcommand{\yhat}{\mathbf{\hat{y}}} % Enhedsvektor y
\newcommand{\zhat}{\mathbf{\hat{z}}} % Enhedsvektor z
\newcommand{\nhat}{\mathbf{\hat{n}}} % Enhedsvektor n
\newcommand{\arbhat}[1]{\mathbf{\hat{#1}}} % Enhedsvektor for arbitrært input
\newcommand{\Grad}[1]{\vgv{\nabla} #1} % Gradient
\let\divsymb=\div % Omdøb \div til \divsymb, så \div{} er en fri kommando
\renewcommand{\div}[1]{\vgv{\nabla} \cdot #1} % Divergens
\newcommand{\Curl}[1]{\vgv{\nabla} \times #1} % Curl
    % Vil man tage div eller curl af græske bogstaver, skal man lade argumentet være f.eks. \vgv{\mu} for µ-vektor
\newcommand{\laplace}[1]{\vgv{\nabla}^2 #1} % Laplaceoperator

% Pæne græske bogstaver
\renewcommand{\epsilon}{\varepsilon}
\renewcommand{\phi}{\varphi}

% Matematiske mængder
\newcommand{\N}{\ensuremath{\mathbb{N}}} % Naturlige tal
\newcommand{\Z}{\ensuremath{\mathbb{Z}}} % Hele tal
\newcommand{\Q}{\ensuremath{\mathbb{Q}}} % Rationelle tal
\newcommand{\R}{\ensuremath{\mathbb{R}}} % Reelle tal
\newcommand{\C}{\ensuremath{\mathbb{C}}} % Komplekse tal
\newcommand{\F}{\ensuremath{\mathbb{F}}} % Legeme tal
\newcommand{\A}{\ensuremath{\mathbb{A}}} % Algebraiske tal

% Belejlige definitioner
\newcommand{\e}[1]{\cdot 10^{#1}} % *10^n ved \e{n}	
\newcommand{\inv}[1]{\dfrac{1}{#1}} % 1/x ved \inv{x}
\newcommand{\invsqrt}[1]{\dfrac{1}{\sqrt{#1}}} % 1/sqrt{x} ved \invsqrt{x}
\newcommand{\invsqrtTo}{\invsqrt{2}} % 1/sqrt{2}
\newcommand{\sqrtTo}{\sqrt{2}} % sqrt{2}
\newcommand{\sqrtTre}{\sqrt{3}} % sqrt{3}
\newcommand{\sqf}[2]{\sqrt{\frac{#1}{#2}}} % \sqrt{\frac{a}{b}} som \sqf{a}{b}
\newcommand{\sqfBot}{\invsqrt} % \frac{1}{\sqrt{a}} som \sqfBot{a}
% \newcommand{\half}{\frac{1}{2}} % \frac{1}{2}

% Redefinering af funktioner, så de tillader størrelsestilpassende parenteser
% \newcommand{\pexp}[1]{\exp\left( #1 \right)} % Eksponentialfunktion
% \newcommand{\bexp}[1]{\exp\left[ #1 \right]} % Eksponentialfunktion
%...
\newcommand{\pfrac}[2]{\left(\frac{#1}{#2}\right)} % Brøker, \pfrac{a}{b} for (a/b)

% Identitet
\usepackage{bbold}
\newcommand{\id}{\mathbb{1}} % Identiteten skrevet som 1 med to steger (som matematiske mængder normalt er)

% Matematisk notation
\newcommand{\dak}{^{\dagger}}	% Hermitisk konjugering skrives "\dak"
\newcommand{\T}{^\intercal} % Transponering som \T
% \DeclareMathOperator\erf{erf}	% Error function as \erf
% \DeclareMathOperator\erfc{erfc}	% Complementary error function as \erfc

% Resten af en sekvens med højereordensled
\newcommand{\Onot}{\mathcal{O}}

% Negation af matematiske tegn
\usepackage{centernot} % Brug \centernot{<command>}